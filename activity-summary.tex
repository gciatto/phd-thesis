% !TeX root = phd-thesis.tex
% !TeX spellcheck = en_GB

\unnumberedchap{Summary of Candidate's Activity}

\section*{Education}

\begin{activity}{PhD Student}{\textbf{November, 2017 $\rightarrow$ January, 2022}}{\theprogramme}{\uniboen, Italy}
    \item Four-years-long PhD programme
	\item Studies in the field of artificial intelligence, focusing on novel enabling approaches, architectures and technologies
    \item Supervisor: Prof. \href{mailto:andrea.omicini@unibo.it}{Andrea Omicini}
\end{activity}

\begin{activity}{Visiting PhD Student}{\textbf{May, 2019 $\rightarrow$ August, 2019}}{University of Applied Sciences Western Switzerland (HES-SO)}{Sierre, Valais, Switzerland}
    \item Studies in the field of eXplanable Artificial Intelligence (XAI) and Multi Agent Systems (MAS)
    \item Reference:  Prof. \href{mailto:michael.schumacher@hevs.ch}{Michael Ignaz Schumacher}
    \item Reference:  Dr. \href{mailto:davide.calvaresi@hevs.ch}{Davide Calvaresi}
\end{activity}

\begin{activity}{PhD Courses Attendance + Exam Pass}{\textbf{Years I--II}}{27 credits in total}{(1 credit $\approx$ 10h)}
    \item ``Infrastructures for Big Data Processing Basic'' (4 credits), held by dr. \href{mailto:davide.salomoni@cnaf.infn.it}{Davide Salomoni}
    %
    \item ``Infrastructures for Big Data Processing Advanced'' (4 credits), held by dr. \href{mailto:davide.salomoni@cnaf.infn.it}{Davide Salomoni}
    %
    \item ``Advanced Machine Learning'' (4 credits), held by prof. \href{mailto:daniele.bonacorsi@bo.infn.it}{Daniele Bonacorsi}
    %
    \item ``Developing, maintaining, and sharing software tools for research'' (2 credits) held by dr. \href{mailto:danilo.pianini@unibo.it}{Danilo Pianini}
    %
    \item ``Blockchain and Cryptocurrencies'' (1 credit) held by prof. \href{mailto:stefano.ferretti@unibo.it}{Stefano Ferretti}
    %
    \item ``Models and Algorithms for Matching and Assignment Problems'' (2 credit) held by Prof. \href{mailto:pierluigi.martelli@unibo.it}{Silvano Martello}
    %
    \item ``Programming for Bioinformatics'' (10 credits) held by prof. \href{mailto:silvano.martello@unibo.it}{Pier Luigi Martelli}
\end{activity}

\begin{activity}{PhD Courses Attendance Only}{\textbf{Years I--II}}{8 credits in total}{(1 credit $\approx$ 10h)}
    \item ``Statistical Learning and Applications'' (6 credits) held by prof.  \href{mailto:sergio.pastorello@unibo.it}{Sergio Pastorello}
    %
    \item ``Introduction to complex systems science'' (2 credits) held by dr. \href{mailto:andrea.roli@unibo.it}{Andrea Roli}
\end{activity}

\begin{activity}{PhD School Attendance}{\textbf{Years I--II}}{15 credits in total}{(1 credit $\approx$ 10h)}
    \item ``Bertinoro International Spring School 2018'' (4 credits), organised by the ``Coordinamento Nazionale dei Dottorati in Informatica'' (\url{http://www.cs.unibo.it/projects/biss2018/index.html})

    \item ``Second International PhD School on Open Science Cloud'' (4 credits), organised by INFN, CNAF, the ``Dipartimento di Fisica e Geologia'' of the University of Perugia, and the ``Dipartimento di Fisica e Astronomia'' of the University of Bologna'' (\url{https://agenda.infn.it/conferenceDisplay.py?confId=15534})

    \item ``First International School on Data Science and IoT'' (3 credits), organised by the ``Dipartimento di Ingegneria Elettrica, Elettronica e Informatica'' of University of Catania, and by the ``Consorzio Cometa'' (\url{http://isdit.dieei.unict.it/ISDIT/Home.html})

    \item ``Summer School on Advances in Artificial Intelligence'' (4 credits), organised by the ``Dipartimento di Informatica, Sistemistica e Comunicazioni'' of the University of Milano-Bicocca, and by the ``Alessandro Volta'' Foundation
     (\url{https://cibr.lakecomoschool.org})
\end{activity}

\begin{activity}{International Conferences and Workshops Attendance}{\textbf{Years I--IV}}{more than 15 credits in total}{(1 credit $\approx$ 10h)}
    \item ``Globe-IoT 2018'' workshop (\url{http://plasma.dimes.unical.it/events/Globe-IoT2018}) co-located with the ``International Conference on Internet-of-Things Design and Implementation'', to present the paper \cite{lpaas-ic2e2018}
    %
    \item ``WOA 2018'' workshop (\url{http://diid.unipa.it/roboticslab/woa2018}) to present the papers \cite{blockchainlp-woa2018,spacetimelp-woa2018}
    %
    \item ``GOODTECHS 2018'' workshop (\url{http://goodtechs2018.eai-conferences.org}) co-located with the ``$4^{th}$ EAI International Conference on Smart Objects and Technologies for Social Good'', to present the paper \cite{blockchain-goodtechs2018}
    %
    \item ``BCT4MAS 2018'' workshop (\url{http://bct4mas.santannapisa.it/}) co-located with the ``EEE/WIC/ACM International Conference on Web Intelligence'', to present the paper \cite{bctcoord-bct4mas2018wi}
    %
    \item ``PAAMS 2019'' international conference (\url{https://edition2019.paams.net/}) on ``Practical Applications of Agents and Multi-Agent Systems'' and its co-located events (BLOCKCHAIN'19 and BCT4MAS 2019), to present the papers \cite{autonomoussc-paams2019,proactivesc-blockchain2019,bctcoord-bct4mas2019}
    %
    \item ``EoT 2019'' workshop (\url{http://plasma.deis.unical.it/events/EoT2019}), co-located with the ``$28^{th}$ International Conference on Computer Communications and Networks'', to present the paper \cite{tusow-icccn2019}
    %
    \item ``AI\&IoT 2019'' workshop, co-located with the ``$18^{th}$ International Conference of the Italian Association for Artificial Intelligence'', to present the paper \cite{xmas-aiiot2019}
    %
    \item ``WOA 2020'' workshop (\url{http://woa2020.apice.unibo.it/}), to present the paper \cite{kotlindsi4prolog-woa2020}
    %
    \item ``AAMS 2020'' international conference (\url{https://aamas2020.conference.auckland.ac.nz/}) on ``Autonomous Agents and Multi-Agent Systems'' and its co-located events (EXTRAAMAS 2020), to present the papers \cite{agentbasedxai-aamas2020,agentbasedxai-extraamas2020}
    %
    \item ``JELIA 2021'' European conference (\url{https://jelia2021.aau.at/}) on ``Logics in Artificial Intelligence '', to present the papers \cite{2pkt-jelia2021}
    %
    \item ``EXTRAAMAS 2021'' workshop (\url{https://extraamas.ehealth.hevs.ch/archive.html}) -- co-located with the ``$20^{th}$ International Conference on Autonomous Agents and Multi-Agent Systems'' (AAMAS 2021) --, as the Publicity Chair and as an author of the papers \cite{imagination-extraamas2021,shallow2deep-extraamas2021,gridex-extraamas2021}
    %
    \item ``WOA 2021'' workshop (\url{http://woa2021.apice.unibo.it/}), as an organiser and an author of the papers \cite{psyke-woa2021,gnn-woa2021}
    %
    \item ``AIxIA 2021'' international conference (\url{https://aixia2021.disco.unimib.it/}) of the ``Italian Association for Artificial Intelligence'', to present the papers \cite{dcc-aixia-2021-plp}
\end{activity}

\section*{Faculty Activity}

\begin{activity}{Students Representative}{\textbf{September, 2019 $\rightarrow$ Now}}{Executive Board of DISI}{\uniboen, Italy}
    \item \url{https://disi.unibo.it/it/dipartimento/organizzazione/organi-di-dipartimento}
\end{activity}

\begin{activity}{Students Representative}{\textbf{July, 2019 $\rightarrow$ Now}}{Council of DISI}{\uniboen, Italy}
    \item \url{https://disi.unibo.it/it/Dipartimento/il-consiglio-di-dipartimento}
\end{activity}

\begin{activity}{PhD students Representative}{\textbf{June, 2019 $\rightarrow$ Now}}{Council of the ``Data Science and Computation' PhD Programme}{\uniboen. Italy}
    \item Reference: Prof. \href{mailto:andrea.cavalli@unibo.it}{Andrea Cavalli}, PhD Programm Coordinator
\end{activity}


\section*{Scientific Activity}

\begin{activity}{Organizing Chair of CILC 2022}{\textbf{June, 2022}}{Blended Workshop}{Bologna, Italy, Jun. 29 -- Jul. 1, 2022}
    \item $37^{th}$ Italian Conference on Computational Logic
    \item \url{http://cilc2022.apice.unibo.it/}
\end{activity}

\begin{activity}{Track chair of EXTRAAMAS 2022}{\textbf{May, 2022}}{Workshop, co-hosted by AAMAS 2022}{Auckland, New Zeland, May 9--13, 2022}
    \item Special Track on ``The chist-ERA of XAI''
    \item ``$4^{th}$ International Workshop on EXplainable and TRAnsparent AI and Multi-Agent Systems
    \item \url{https://extraamas.ehealth.hevs.ch/}
\end{activity}

\begin{activity}{Program Committee Membership for EAAI 2022}{\textbf{November, 2021}}{Workshop, co-hosted by AAAI 2022}{Vancouver, Canada, Feb. 28 -- Mar. 1, 2022}
    \item ``Explainable Agency in Artificial Intelligence'' Workshop
    \item \url{https://sites.google.com/view/eaai-ws-2022/organization}
\end{activity}

\begin{activity}{Guest Editor of IA 2021}{\textbf{September, 2021}}{WOA Special Issue}{}
    \item ``Intelligenza Artificiale'' Journal
    \item \url{https://www.iospress.com/catalog/journals/intelligenza-artificiale}
\end{activity}

\begin{activity}{Organizing Chair of WOA 2021}{\textbf{September, 2021}}{Blended Workshop}{Bologna, Italy, Sept. 1--3, 2021}
    \item $22^{th}$ Workshop ``From Objects to Agents''
    \item \url{http://ceur-ws.org/Vol-2963}
    \item \url{http://woa2021.apice.unibo.it/}
\end{activity}

\begin{activity}{Program Committee Membership for ICHMS 2021}{\textbf{June, 2021}}{Virtual Conference}{Magdeburg, Germany, Sept. 8--10, 2021}
    \item 2021 IEEE ``International Conference on Human-Machine Systems''
    \item \url{https://www.ichms2021.de/#committee}
\end{activity}

\begin{activity}{Publicity Chair of EXTRAAMAS 2021}{\textbf{May, 2021}}{Virtual Workshop, co-hosted by AAMAS 2021}{London, UK, May 3--7, 2021}
    \item $21^{th}$ International Workshop on EXplainable and TRAnsparent AI and Multi-Agent Systems
    \item \url{https://link.springer.com/book/10.1007%2F978-3-030-82017-6}
    \item \url{https://extraamas.ehealth.hevs.ch/}
\end{activity}

\begin{activity}{Organizing Chair of WOA 2020}{\textbf{September, 2020}}{Virtual Workshop}{Bologna, Italy, Sept. 14--16, 2020}
    \item $21^{th}$ Workshop ``From Objects to Agents''
    \item \url{http://ceur-ws.org/Vol-2706}
    \item \url{http://woa2020.apice.unibo.it/}
\end{activity}

\begin{activity}{Lecturer at WOA 2018 Doctoral School}{\textbf{June 27, 2018}}{19th Workshop From Objects to Agents (WOA)}{Palermo, Italy}
    \item Talk title: ``Blockchain \& Smart Contracts: Basics and Perspectives for MAS''
    \item Reference: \url{http://diid.unipa.it/roboticslab/woa2018/index.php/mini-school}
\end{activity}

\section*{Project Management Activity}

\subsection*{\textsc{Expectation} (G.A. \texttt{CHIST-ERA-19-XAI-005})}

\begin{activity}{Work Package Leader}{\textbf{April 2021 $\rightarrow$ Ongoing}}{WP2 -- Modelling INTRA-Agent explainability}{}
    \item Principal Investigator: Prof. \href{mailto:michael.schumacher@hevs.ch}{Michael Ignaz Schumacher}
    \item Italian Partners' Scientific \& Technical Coordinator: Prof. \href{mailto:andrea.omicini@unibo.it}{Andrea Omicini}
    \item Project Web Site: \url{https://expectation.ehealth.hevs.ch}
\end{activity}

\section*{Teaching Activity}

\subsection*{University, $2^{nd}$ cycle (Master's Degree Courses)}

\begin{activity}{Teaching assistant}{\textbf{September $\rightarrow$ December, 2017--2021}}{``Distributed Systems'' $2^{nd}$ cycle course}{\uniboen, Italy}
    \item Distributed architectures, ReSTfull Web-Services, Containers,  Agent-based technologies and middlewares, Cloud computing, Blockchain technologies
    \item Main teacher: Prof. \href{mailto:andrea.omicini@unibo.it}{Andrea Omicini}
    \item Course info: \url{https://apice.unibo.it/xwiki/bin/view/Courses/Sd2122}
\end{activity}

\subsection*{University, $1^{st}$ cycle (Bachelor's Degree Courses)}

\begin{activity}{Teaching assistant}{\textbf{September $\rightarrow$ December, 2017/2018}}{``Object oriented programming'' $1^{st}$ cycle course}{\uniboen, Italy}
    \item Foundations of the object oriented programming paradigm, design patterns, concurrency basics, and GUI development in Java + C\# and .NET basics
    \item Man teacher: Prof. \href{mailto:mirko.viroli@unibo.it}{Mirko Viroli}
    \item Course info: \url{http://apice.unibo.it/xwiki/bin/view/Courses/Sd1819}
\end{activity}

\subsection*{Masters}

\begin{activity}{Lecturer at Bologna Business School}{\textbf{November 14-15, 2018}}{Amadori Graduate Program}{Bologna, Italy}
	\item Talk title: ``Blockchain \& Smart Contracts: What are they? Do we need them?''
\end{activity}

\subsection*{Industry}

\begin{activity}{Teacher at IMA S.p.A.}{\textbf{April $\rightarrow$ June, 2021}}{``ENG19 -- Advanced OO Programming in C\#'' course}{Bologna, Italy}
    \item Test-driven development in .NET, OOP Design patterns, domain driven design, concurrency and multithreading in .NET
    \item IMA Web site: \url{https://ima.it}
    \item Reference: \href{mailto:formazione@ima.it}{formazione@ima.it}
\end{activity}

\begin{activity}{Teacher at IMA}{\textbf{March $\rightarrow$ April, 2021}}{``ENG18 -- Basic OO Programming in C\#'' course}{Bologna, Italy}
    \item .NET solution and project management, collections and exceptions in .NET, advanced mechanisms of C\#
    \item IMA Web site: \url{https://ima.it}
    \item Reference: \href{mailto:formazione@ima.it}{formazione@ima.it}
\end{activity}

\begin{activity}{Teacher at professional education course}{2019/2020}{IFTS course by \href{http://www.formart.it/home}{FORMart}}{Cesena, Italy}
    \item Talks topics: Cloud Computing and Blockchain Technologies
    \item References:  Prof. \href{mailto:a.ricci@unibo.it}{Alessandro Ricci}
\end{activity}

%----------------------------------------------------------------------------------------
%	SOFTWARE DEVELOPMENT
%----------------------------------------------------------------------------------------

\section*{Development of Research-related Software}

\begin{activity}{\textsf{tu}Prolog (2P) \cite{cco-softwarex-2021-2pkt}}{\textbf{April, 2019 $\rightarrow$ Ongoing}}{}{}
    \item A logic programming framework supporting multi-paradigm programming via a clean, seamless, and bidirectional integration between the logic and object-oriented paradigms
    \item \url{http://tuprolog.unibo.it}
    \item \url{https://github.com/tuProlog/2p-kt}
\end{activity}

\begin{activity}{\textsc{TuSoW} \cite{tusow-icccn2019}}{\textbf{December, 2018 $\rightarrow$ Ongoing}}{}{}
    \item Tuple Spaces over the Web: a framework for the coordination of distributed software agents via \textsc{Linda}-like tuple spaces
    \item \url{https://github.com/CoordaaS/TuSoW}
\end{activity}

\begin{activity}{\textsc{2ppy} (2P in Python)}{\textbf{September, 2021 $\rightarrow$ Ongoing}}{}{}
    \item Porting of 2P on Python
    \item \url{https://github.com/tuProlog/2ppy}
\end{activity}

\begin{activity}{\textsc{Psyke} \cite{psyke-woa2021}}{\textbf{October, 2021 $\rightarrow$ Ongoing}}{}{}
    \item Platform for Symbolic Knowledge Extraction (in the form of logic rules) our of sub-symbolic predictors
    \item \url{https://github.com/psykei/psyke-python}
\end{activity}
