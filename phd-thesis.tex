\documentclass[12pt,a4paper,openright,twoside]{book}
\usepackage[utf8]{inputenc}
\usepackage{phd-thesis}

\school{ALMA MATER STUDIORUM -- UNIVERSITÀ DI BOLOGNA}
\programme{Dottorato di Ricerca in Data Science and Computation}
\title{On the role of Computational Logics in modern Data Science: representing, learning, reasoning, and explaining knowledge}
\author{Giovanni Ciatto}
\date{\today}
\contestsector{09/H1 -- Sistemi di Elaborazione delle Informazioni}
\scientificsector{ING-INF/05 -- Sistemi di Elaborazione delle Informazioni}
\coordinator{Andrea Cavalli}
\supervisor{Andrea Omicini}
\cycle{XXXIII}
\examyear{2022}

\begin{document}
	
\frontmatter

% ! TeX root = thesis-main.tex
\title{Title}
\author{Candidate Name Here}
\date{\today}

\newgeometry{margin=0.8in}
\begin{titlepage}
	\begin{center}
		% \vspace*{0.2cm}
		
		\large
		\textbf{ALMA MATER STUDIORUM -- UNIVERSITÀ DI BOLOGNA}
		\\
		\noindent\hrulefill
		\vspace{0.4cm}
		
		\Large
		Dottorato di Ricerca in
		\\
		Data Science and Computation

		\Huge
		\vspace{4cm}
		\textbf{
			On the role of Computational Logics in modern Data Science: representing, learning, reasoning, and explaining knowledge
		}
		
		\large
		\vspace{1cm}
		Settore Concorsuale:
		\\
		\textsc{09/H1 -- Sistemi di Elaborazione delle Informazioni}
		\\\vspace{0.5cm}
		Settore Scientifico Disciplinare:
		\\
		\textsc{ING-INF/05 -- Sistemi di Elaborazione delle Informazioni}

		
		\vspace{4cm}
		%
		\begin{minipage}[t]{0.64\textwidth}
			\begin{flushleft}
				\textit{Coordinatore Dottorato} 
				\\ 
				\textbf{Prof.} \textbf{Andrea Cavalli}
				\\
				\vspace{0.4cm}
				\textit{Supervisore} 
				\\
				\textbf{Prof.} \textbf{Andrea Omicini}
			\end{flushleft}
		\end{minipage}
		%
		\begin{minipage}[t]{0.34\textwidth}
			\begin{flushright}
				\textit{Candidato} 
				\\ 
				\textbf{Giovanni Ciatto}
			\end{flushright}
		\end{minipage}
		\\
		
		\vfill
		\noindent\hrulefill
		\vspace{0.3cm}
		\Large
		
		XXXIII Ciclo
		\\
		Esame Finale Anno 2022
	\end{center}
\end{titlepage}
\restoregeometry


\begin{abstract}	
Max 2000 characters, strict.
\end{abstract}

\begin{dedication} % this is optional
Optional. Max a few lines.
\end{dedication}

\begin{acknowledgements} % this is optional
Optional. Max 1 page.
\end{acknowledgements}

%----------------------------------------------------------------------------------------
\tableofcontents   
\listoffigures     % (optional) comment if empty
\lstlistoflistings % (optional) comment if empty
%----------------------------------------------------------------------------------------

\mainmatter

%----------------------------------------------------------------------------------------
\chapter{\introductionname}
\label{chap:introduction}
%----------------------------------------------------------------------------------------

In the last decade we have witnessed an explosion in the exploitation of artificial intelligence (AI) both in the academy and in the industry, and in virtually all strategical sectors of human expertise.
%
This is not the first time in history that AI attains unprecedented levels of attention, expectation, and fundings, yet it is the first time that such momentum is driven by a pervasive adoption of data science (DS) and, in particular, machine learning (ML).

Nowadays, the tree terms -- AI, DS, and ML -- are often used mistakenly interchangeably, especially by practitioners.
%
Should we speculate on what the causes of such phenomenon are, we would argue this is likely due to the strong hype characterising modern data-driven solutions---both in theory and in practice.
%
This leads both researchers and practitioners to focus on the development of \emph{ML-oriented} frameworks or technologies which, in turn, create a sampling bias making people think that ML exhausts DS, and DS saturates AI.
%
As we further discuss in the subsequent chapters, this is really far from the truth.
%
There are many interesting aspects of AI which lay outside the realm of DS.
%
Notably, in this thesis we focus on computational logics (CL) -- a prominent aspect of AI populating the portion which is not covered by DS -- and its potential role in complementing DS.

As sub-fields of AI, both DS and CL share the common goal of mimicking human intelligence.
%
Of course, they do so in different ways.
%
They focus on different notions and aspects of intelligence, they pursue intelligence through different ways, and for different purposes.
%
Notably, most differences lay in the way CL and DS treat \emph{knowledge}, and, in particular, in the way knowledge is represented, acquired, manipulated, and transferred.

CL, for instance, focuses on \emph{rational} intelligence, and it aims at endowing machines with human-like, automated \emph{reasoning} capabilities.
%
Following this purpose, it relies on \emph{symbolically} represented knowledge, either acquired via logic induction or via manual handcrafting, manipulated via logic inference (e.g. deduction or abduction), and transferred by simply presenting symbols into shared formats.
%
Dually, DS focuses on \emph{intuitive} intelligence, and it aims at endowing humans with statistical tools for mining significant and predictive information from data, in a principled way.
%
When applied to machines, DS provides them with powerful pattern matching, recognition, or stimulus-response capabilities. 
%
For this reason, it relies on sub-symbolically (e.g. \emph{numerically}) represented knowledge, commonly acquired from data via ML, manipulated via algebraic or differential operations, and transferred in disparate, purpose-specific ways.

Of course, both CL and DS come with shortcomings.
%
On the one side, CL commonly requires
%
\begin{inlinelist}
    \item some symbolic knowledge to be eventually handcrafted by humans, manually; and
    \item the task at hand to have a clear formulation, which can be expressed via crisp symbols.
\end{inlinelist}
%
The former issue, clearly hinders scalability, making CL fall short on the knowledge provisioning side.
%
Vice versa, DS is very well suited on this side, as it naturally leverages on scalable algorithms which have been designed mine information semi-automatically from data, possibly scaling up to very large datasets.
%
The latter issue, in turn, makes CL poorly suited to handle fuzzy tasks which are hard to formalise or encode symbolically---think, for instance, to the task of handwritten digits recognition.
%
On the other side, DS commonly requires
%
\begin{inlinelist}
    \item very large amounts of data to be effective; and
    \item users to be willing and capable of interpreting the numeric results it outputs.
\end{inlinelist}
%
The former issue actually constrains the exploitation of DS into use cases where data is already available or a provisioning procedure is admissible.
%
Vice versa, CL is data efficient as it can bring valuable results even in presence of very small prior knowledge.
%
The latter issue is, in turn, among the most relevant topic nowadays.
%
Given the wide exploitation of DS in some many areas of expertise, clarity and intelligibility of its outcomes are becoming a critical aspects---mostly because of their sub-symbolic nature.
%
Vice versa, CL is inherently symbolic in nature and therefore less subject to such interpretability issues.

Accordingly, this thesis stems from the acknowledgement that CS and DS are complementary rather than in competition.
%
Along this line, it aims to \emph{elicit} and \emph{enable} the many possible bridges among the two fields.

On the one side, we \emph{elicit} analogies, dichotomies, and possible synergies among CL and DS by analysing them along four orthogonal dimensions, corresponding to as many knowledge-related activities, namely:
%
\begin{description}
    \item[representation] | i.e. the way knowledge is expressed and made interpretable by either machines or human beings, or both; e.g. via symbols, formul\ae, or tensors of real numbers 
    \item[acquisition] | i.e. the way novel knowledge is learned from prior information, mined from data, or attained from external sources; e.g. via data mining, via induction, or via interaction 
    \item[inference] |  i.e. the way decisions, suggestions, recommendations, or predictions can be automatically computed out of prior knowledge; e.g. via automated deduction/abduction, or via classification/regression
    \item[explanation] | i.e. the way knowledge can be transferred to another entity---be it computational or human
\end{description}

On the other side, we acknowledge that both CL and DS have a prominent overlap with computer science (CS) and software engineering (SE).
%
Regardless of how they manipulate knowledge, both approaches subtend a mathematical modelling of many computational aspects, which must then be reified into well-engineered software technologies to let practitioners actually exploit them. 
%
Accordingly, we further analyse CL and DS from both a computational and technological perspective.
%
While the computational perspective focuses on \emph{what} data structures, algorithms, and workflows they leverage upon to attain intelligence, the technological perspective focuses on \emph{how} such aspects can be translated in practice, via robust software architectures and effective implementations.
%
Along this line, in particular, we assess the current state of the art for technologies laying at the intersection among DS and CL -- or supporting the construction of bridges among the two fields --, identifying holes and proposing lacks to overcome them.
%
The latter in particular is the contribution by which we \emph{enable} the actual combination of CL and DS in practice.

\gcnote{Agent oriented mindset.}

\part{What}
\label{part:what}

\chapter{Symbolic and Sub-Symbolic sides of AI}

\section{Computational Logic}

\cite{cco-softwarex-2021-2pkt}
\cite{Korner2020HistoryFuturePrologTPLP}

\section{Data Science and Machine Learning}

\cite{xailp-woa2019}
\cite{xaisurvey-ia14}

\chapter{Representing Data and Knowledge}

\section{Distributed vs. Symbolic}

\section{Intensional vs. Extensional}

\section{Relational vs. Functional}

\chapter{Learning Knowledge from Data}

\section{Machine Learning}

\section{Logic Induction}

\chapter{Generating Data by Reasoning over Knowledge}

\section{Symbolic Reasoning}

\subsection{Inference}

Deduction, Abduction, Induction, Probabilistic, etc.

\subsection{Logic Programming}

\cite{logictech-information11}
\cite{lptech4mas-aamas2021}
\cite{lptech4mas-jaamas35}
\cite{Korner2020HistoryFuturePrologTPLP}

\section{Sub-symbolic Reasoning}

\subsection{Neuro-Symbolic Computation}

\subsection{Knowledge Graph Embedding}

\chapter{Explaining AI via Symbolic Knowledge}

\cite{ccnavos-extraamas2021-expectation}

\section{Explanation vs. Interpretation: Overview}

\cite{agentbasedxai-aamas2020}
\cite{agentbasedxai-extraamas2020}

\section{Symbolic Knowledge Extraction}

\cite{aco-extraamas2021-shallow2deep}
\cite{xailp-woa2019}

\section{Symbolic Knowledge Injection}

\cite{nsc4xai-woa2020}

\part{How}
\label{part:how}

\chapter{The Role of Software}

\chapter{Technological State of the Art}

\cite{coordination-jlamp2020}

\section{Current State of Logic-Based Technologies}

\cite{lptech4mas-aamas2021}
\cite{lptech4mas-jaamas35}
\cite{logictech-information11}

\section{Current State of Machine Learning Technologies}

\section{Current State of XAI Technologies}

\cite{xaisurvey-ia14}

\chapter{Need for An Open Ecosystem for Logic-Based AI}

\cite{cco-softwarex-2021-2pkt}

\chapter{The 2P-Kt Ecosystem}

\cite{cco-softwarex-2021-2pkt}
\cite{kotlindsi4prolog-woa2020}

\chapter{Bridging Logic Programming and Data Processing}

\cite{2pkt-jelia2021}

\chapter{Bridging Logic Programming and Object Orientation}

\cite{cco-softwarex-2021-2pkt}
\cite{kotlindsi4prolog-woa2020}

\chapter{Bridging Logic Programming and Machine Learning}

Castiglio

\chapter{Bridging Logic Programming and XAI}

Psyke

\chapter{Enriching the Ecosystem}

\section{Probabilistic Logic Programming}

Jason

\section{Argumentation}

Pisano

\section{Inductive Logic Programming}

Speciale

\part{Who}
\label{part:who}

\cite{cncc-extraamas2021-imagination}
\cite{ccnavos-extraamas2021-expectation}

\chapter{Adding Control to Data via Agents}

\chapter{On the role of Interaction}

\cite{tusow-icccn2019}
\cite{respect-idc2017}
\cite{respectx-comsis15}

\chapter{Blockchain as the way to Trustworthiness}

\cite{bctcoord-bct4mas2018wi}
\cite{bctcoord-bct4mas2019}
\cite{bctcoordination-information11}
\cite{blockchain-goodtechs2018}
\cite{proactivesc-blockchain2019}
\cite{blockchainmas-applsci10}


%----------------------------------------------------------------------------------------
% BIBLIOGRAPHY
%----------------------------------------------------------------------------------------

\nocite{*} % uncomment this to show all the reference in the .bib file
\bibliographystyle{plain}
\bibliography{phd-thesis}


\end{document}