\documentclass[12pt,a4paper,openright,twoside]{book}
\usepackage[utf8]{inputenc}
\usepackage{phd-thesis}

\school{ALMA MATER STUDIORUM -- UNIVERSITÀ DI BOLOGNA}
\programme{Dottorato di Ricerca in Data Science and Computation}
\title{On the role of Computational Logics in modern Data Science: representing, learning, reasoning, and explaining knowledge}
\author{Giovanni Ciatto}
\date{\today}
\contestsector{09/H1 -- Sistemi di Elaborazione delle Informazioni}
\scientificsector{ING-INF/05 -- Sistemi di Elaborazione delle Informazioni}
\coordinator{Andrea Cavalli}
\supervisor{Andrea Omicini}
\cycle{XXXIII}
\examyear{2022}

\begin{document}
	
\frontmatter

% ! TeX root = thesis-main.tex
\title{Title}
\author{Candidate Name Here}
\date{\today}

\newgeometry{margin=0.8in}
\begin{titlepage}
	\begin{center}
		% \vspace*{0.2cm}
		
		\large
		\textbf{ALMA MATER STUDIORUM -- UNIVERSITÀ DI BOLOGNA}
		\\
		\noindent\hrulefill
		\vspace{0.4cm}
		
		\Large
		Dottorato di Ricerca in
		\\
		Data Science and Computation

		\Huge
		\vspace{4cm}
		\textbf{
			On the role of Computational Logics in modern Data Science: representing, learning, reasoning, and explaining knowledge
		}
		
		\large
		\vspace{1cm}
		Settore Concorsuale:
		\\
		\textsc{09/H1 -- Sistemi di Elaborazione delle Informazioni}
		\\\vspace{0.5cm}
		Settore Scientifico Disciplinare:
		\\
		\textsc{ING-INF/05 -- Sistemi di Elaborazione delle Informazioni}

		
		\vspace{4cm}
		%
		\begin{minipage}[t]{0.64\textwidth}
			\begin{flushleft}
				\textit{Coordinatore Dottorato} 
				\\ 
				\textbf{Prof.} \textbf{Andrea Cavalli}
				\\
				\vspace{0.4cm}
				\textit{Supervisore} 
				\\
				\textbf{Prof.} \textbf{Andrea Omicini}
			\end{flushleft}
		\end{minipage}
		%
		\begin{minipage}[t]{0.34\textwidth}
			\begin{flushright}
				\textit{Candidato} 
				\\ 
				\textbf{Giovanni Ciatto}
			\end{flushright}
		\end{minipage}
		\\
		
		\vfill
		\noindent\hrulefill
		\vspace{0.3cm}
		\Large
		
		XXXIII Ciclo
		\\
		Esame Finale Anno 2022
	\end{center}
\end{titlepage}
\restoregeometry


\begin{abstract}	
Max 2000 characters, strict.
\end{abstract}

\begin{dedication} % this is optional
Optional. Max a few lines.
\end{dedication}

\begin{acknowledgements} % this is optional
Optional. Max 1 page.
\end{acknowledgements}

%----------------------------------------------------------------------------------------
\tableofcontents   
\listoffigures     % (optional) comment if empty
\lstlistoflistings % (optional) comment if empty
%----------------------------------------------------------------------------------------

\mainmatter

%----------------------------------------------------------------------------------------
\chapter{\introductionname}
\label{chap:introduction}
%----------------------------------------------------------------------------------------

\part{What}
\label{part:what}

\chapter{Symbolic and Sub-Symbolic sides of AI}

\section{Computational Logic}

\cite{cco-softwarex-2021-2pkt}
\cite{Korner2020HistoryFuturePrologTPLP}

\section{Data Science and Machine Learning}

\cite{xailp-woa2019}
\cite{xaisurvey-ia14}

\chapter{Representing Data and Knowledge}

\section{Distributed vs. Symbolic}

\section{Intensional vs. Extensional}

\section{Relational vs. Functional}

\chapter{Learning Knowledge from Data}

\section{Machine Learning}

\section{Logic Induction}

\chapter{Generating Data by Reasoning over Knowledge}

\section{Symbolic Reasoning}

\subsection{Inference}

Deduction, Abduction, Induction, Probabilistic, etc.

\subsection{Logic Programming}

\cite{logictech-information11}
\cite{lptech4mas-aamas2021}
\cite{lptech4mas-jaamas35}
\cite{Korner2020HistoryFuturePrologTPLP}

\section{Sub-symbolic Reasoning}

\subsection{Neuro-Symbolic Computation}

\subsection{Knowledge Graph Embedding}

\chapter{Explaining AI via Symbolic Knowledge}

\cite{ccnavos-extraamas2021-expectation}

\section{Explanation vs. Interpretation: Overview}

\cite{agentbasedxai-aamas2020}
\cite{agentbasedxai-extraamas2020}

\section{Symbolic Knowledge Extraction}

\cite{aco-extraamas2021-shallow2deep}
\cite{xailp-woa2019}

\section{Symbolic Knowledge Injection}

\cite{nsc4xai-woa2020}

\part{How}
\label{part:how}

\chapter{The Role of Software}

\chapter{Technological State of the Art}

\cite{coordination-jlamp2020}

\section{Current State of Logic-Based Technologies}

\cite{lptech4mas-aamas2021}
\cite{lptech4mas-jaamas35}
\cite{logictech-information11}

\section{Current State of Machine Learning Technologies}

\section{Current State of XAI Technologies}

\cite{xaisurvey-ia14}

\chapter{Need for An Open Ecosystem for Logic-Based AI}

\cite{cco-softwarex-2021-2pkt}

\chapter{The 2P-Kt Ecosystem}

\cite{cco-softwarex-2021-2pkt}
\cite{kotlindsi4prolog-woa2020}

\chapter{Bridging Logic Programming and Data Processing}

\cite{2pkt-jelia2021}

\chapter{Bridging Logic Programming and Object Orientation}

\cite{cco-softwarex-2021-2pkt}
\cite{kotlindsi4prolog-woa2020}

\chapter{Bridging Logic Programming and Machine Learning}

Castiglio

\chapter{Bridging Logic Programming and XAI}

Psyke

\chapter{Enriching the Ecosystem}

\section{Probabilistic Logic Programming}

Jason

\section{Argumentation}

Pisano

\section{Inductive Logic Programming}

Speciale

\part{Who}
\label{part:who}

\cite{cncc-extraamas2021-imagination}
\cite{ccnavos-extraamas2021-expectation}

\chapter{Adding Control to Data via Agents}

\chapter{On the role of Interaction}

\cite{tusow-icccn2019}
\cite{respect-idc2017}
\cite{respectx-comsis15}

\chapter{Blockchain as the way to Trustworthiness}

\cite{bctcoord-bct4mas2018wi}
\cite{bctcoord-bct4mas2019}
\cite{bctcoordination-information11}
\cite{blockchain-goodtechs2018}
\cite{proactivesc-blockchain2019}
\cite{blockchainmas-applsci10}


%----------------------------------------------------------------------------------------
% BIBLIOGRAPHY
%----------------------------------------------------------------------------------------

\nocite{*} % uncomment this to show all the reference in the .bib file
\bibliographystyle{plain}
\bibliography{phd-thesis}


\end{document}