\documentclass[12pt,a4paper,openright,twoside]{book}
\usepackage[utf8]{inputenc}
\usepackage{phd-thesis}
\usepackage{code-lstlistings}
\usepackage{notes}
\usepackage{shortcuts}

\school{ALMA MATER STUDIORUM -- UNIVERSITÀ DI BOLOGNA}
\programme{Dottorato di Ricerca in Data Science and Computation}
\title{On the role of Computational Logics in modern Data Science: representing, learning, reasoning, and explaining knowledge}
\author{Giovanni Ciatto}
\date{\today}
\contestsector{09/H1 -- Sistemi di Elaborazione delle Informazioni}
\scientificsector{ING-INF/05 -- Sistemi di Elaborazione delle Informazioni}
\coordinator{Andrea Cavalli}
\supervisor{Andrea Omicini}
\cycle{XXXIII}
\examyear{2022}

\mainlinespacing{1.241} % line spacing in mainmatter, comment to default (1)

\begin{document}
	
\frontmatter
\frontispiece

\begin{abstract}	
Max 2000 characters, strict.
\end{abstract}

\begin{dedication} % this is optional
Optional. Max a few lines.
\end{dedication}

\begin{acknowledgements} % this is optional
Optional. Max 1 page.
\end{acknowledgements}

%----------------------------------------------------------------------------------------
\tableofcontents   
\listoffigures     % (optional) comment if empty
\lstlistoflistings % (optional) comment if empty
%----------------------------------------------------------------------------------------

\mainmatter

%----------------------------------------------------------------------------------------
\chapter{Introduction}
\label{chap:introduction}
%----------------------------------------------------------------------------------------

In the last decade we have witnessed an explosion in the exploitation of artificial intelligence (AI) both in the academy and in the industry, and in virtually all strategical sectors of human expertise.
%
This is not the first time in history that AI attains unprecedented levels of attention, expectation, and fundings, yet it is the first time that such momentum is driven by a pervasive adoption of data science (DS) and, in particular, machine learning (ML).

Nowadays, the tree terms -- AI, DS, and ML -- are often used mistakenly interchangeably, especially by practitioners.
%
Should we speculate on what the causes of such phenomenon are, we would argue this is likely due to the strong hype characterising modern data-driven solutions---both in theory and in practice.
%
This leads both researchers and practitioners to focus on the development of \emph{ML-oriented} frameworks or technologies which, in turn, create a sampling bias making people think that ML exhausts DS, and DS saturates AI.
%
As we further discuss in the subsequent chapters, this is really far from the truth.
%
There are many interesting aspects of AI which lay outside the realm of DS.
%
Notably, in this thesis we focus on computational logics (CL) -- a prominent aspect of AI populating the portion which is not covered by DS -- and its potential role in complementing DS.

As sub-fields of AI, both DS and CL share the common goal of mimicking human intelligence.
%
Of course, they do so in different ways.
%
They focus on different notions and aspects of intelligence, they pursue intelligence through different ways, and for different purposes.
%
Notably, most differences lay in the way CL and DS treat \emph{knowledge}, and, in particular, in the way knowledge is represented, acquired, manipulated, and transferred.

CL, for instance, focuses on \emph{rational} intelligence, and it aims at endowing machines with human-like, automated \emph{reasoning} capabilities.
%
Following this purpose, it relies on \emph{symbolically} represented knowledge, either acquired via logic induction or via manual handcrafting, manipulated via logic inference (e.g. deduction or abduction), and transferred by simply presenting symbols into shared formats.
%
Dually, DS focuses on \emph{intuitive} intelligence, and it aims at endowing humans with statistical tools for mining significant and predictive information from data, in a principled way.
%
When applied to machines, DS provides them with powerful pattern matching, recognition, or stimulus-response capabilities. 
%
For this reason, it relies on sub-symbolically (e.g. \emph{numerically}) represented knowledge, commonly acquired from data via ML, manipulated via algebraic or differential operations, and transferred in disparate, purpose-specific ways.

Of course, both CL and DS come with shortcomings.
%
On the one side, CL commonly requires
%
\begin{inlinelist}
    \item some symbolic knowledge to be eventually handcrafted by humans, manually; and
    \item the task at hand to have a clear formulation, which can be expressed via crisp symbols.
\end{inlinelist}
%
The former issue, clearly hinders scalability, making CL fall short on the knowledge provisioning side.
%
Vice versa, DS is very well suited on this side, as it naturally leverages on scalable algorithms which have been designed mine information semi-automatically from data, possibly scaling up to very large datasets.
%
The latter issue, in turn, makes CL poorly suited to handle fuzzy tasks which are hard to formalise or encode symbolically---think, for instance, to the task of handwritten digits recognition.
%
On the other side, to be effective, DS commonly requires
%
\begin{inlinelist}
    \item very large amounts of data; and
    \item users to be willing and capable of interpreting the numeric results it outputs.
\end{inlinelist}
%
The former issue actually constrains the exploitation of DS into use cases where data is already available or a provisioning procedure is admissible.
%
Vice versa, CL is data efficient as it can bring valuable results even in presence of very small prior knowledge.
%
The interpretability issue is, in turn, among the most relevant topic nowadays.
%
Given the wide exploitation of DS in some many areas of expertise, clarity and intelligibility of its outcomes are becoming a critical aspects---mostly because of their sub-symbolic nature.
%
Vice versa, CL is inherently symbolic in nature and therefore less subject to such interpretability issues.

Accordingly, this thesis stems from the acknowledgement that CS and DS are complementary -- rather than competing -- aspects of AI, and that \emph{knowledge} plays a pivotal role in both these fields.
%
Along this line, we aim to \emph{elicit} and \emph{enable} the many possible bridges among them, w.r.t. knowledge manipualtion.
%
In doing so, we follow the ultimate purpose of increasing the degree of intelligence and autonomy of modern computational systems.
%
Therefore, our focus is on computational entities and on the ways they can combine and integrate CL and DS to either act more intelligently or more autonomosly---or both. 

On the one side, we \emph{elicit} analogies, dichotomies, and possible synergies among CL and DS by analysing them along four orthogonal dimensions, corresponding to as many knowledge-related activities, namely:
%
\begin{description}
    \item[representation] | i.e. the way knowledge is expressed and made interpretable by either machines or human beings, or both; e.g. via symbols, formul\ae, or tensors of real numbers 

    \item[acquisition] | i.e. the way novel knowledge is learned from prior information, mined from data, or attained from external sources; e.g. via data mining, via induction, or via interaction 
    
    \item[inference] |  i.e. the way decisions, suggestions, recommendations, or predictions can be automatically computed out of prior knowledge; e.g. via automated deduction/abduction, or via classification/regression
    
    \item[explanation] | i.e. the way knowledge can be transferred to another entity---be it computational or human
\end{description}

On the other side, we acknowledge that both CL and DS have a prominent overlap with computer science (CS) and software engineering (SE).
%
Regardless of how they manipulate knowledge, both approaches subtend a mathematical modelling of many computational aspects, which must then be reified into well-engineered software technologies to let practitioners actually exploit them. 
%
Accordingly, we further analyse CL and DS from both a computational and technological perspective.
%
While the computational perspective focuses on \emph{what} data structures, algorithms, and workflows they leverage upon to attain intelligence, the technological perspective focuses on \emph{how} such aspects can be translated in practice, via robust software architectures and effective implementations.
%
Along this line, in particular, we assess the current state of the art for technologies laying at the intersection among DS and CL -- or supporting the construction of bridges among the two fields --, identifying holes and proposing lacks to overcome them.
%
The latter in particular is the contribution by which we \emph{enable} the actual combination of CL and DS in practice.

We carry out the whole discussion under an agent-oriented mindset.
%
Within the scope of this document, we call ``agent'' any autonomous entity having its own \emph{locus of control}---be it a human being or a running process programmed software.
%
We may refer to agents as ``intelligent'' in case the come equipped with human-like knowledge-related capabilities, such as the aforementioned capabilities of representing, learning, inferring, or explaining knowledge---or, possibly, a multitude of them.
%
Under such a mindset, human agents are intelligent by definition, whereas software agents may tend to intelligence by acquiring one or more of these capabilities via either CS or DS---or, hopefully, a combination of them.
%
Accordingly, what we have so far called ``machines'' are indeed ``software agents'', and the overall role of the agent-oriented mindset is about focussing on \emph{who} is charge of manipulating knowledge and \emph{when}, other than \emph{where} should knowledge be located in the meanwhile.


\paragraph{Structure of the thesis}

Summarising, the whole thesis discusses in what ways CL and DS can jointly contribute to the management of knowledge within the scope of modern and future intelligent systems, and how technically sound software technologies can be realised along the path.
%
An agent-oriented mindset permeates the whole discussion, by stressing pivotal role of autonomous agents in exploiting both means to reach higher degrees of intelligence.
%
Accordingly, this thesis is organised in three parts, named \emph{What}, \emph{How}, and \emph{Who}, respectively.

In the first part (\emph{What}), we focus on the computational perspective.
%
There, we present the two main branches of AI---namely, the \emph{symbolic} and \emph{sub-symbolic} ones.
%
In particular, we recall classical definitions and survey the current state of the art for both CL and DS.
%
We finally present their strengths and weaknesses, and the many possible bridges among them, w.r.t. knowledge representation, acquisition, manipulation, and transfer.

In the second part (\emph{How}), we focus on the technological perspective.
%
There, we survey the current state of technologies.
%
In particular, we focus on logic-based technologies laying at the intersection with DS, identifying the current lacks w.r.t. current theoretical advances. 
%
We then propose the notion of AI ecosystem, and \twopkt{} as its main reification in software.
%
Finally, we present a number of possible ways to extend the \twopkt{} ecosystem towards DS.

Finally, the third part (\emph{Who}) closes the loop by discussing the role of agents \ldots
\note{TBD}


\part{What}
\label{part:what}

\chapter{Symbolic and Sub-Symbolic sides of AI}

AI is a multi-faceted discipline leveraging on contributions coming from several areas of human knowledge, there including Mathematics, Computer Science, Statistics, Psychology, Philosophy, and many others.
%
A famous and comprehensive survey on the many aspects of AI is proposed by Russell and Norving in \cite{russell2016artificial}.
%
In the second half the $20^{th}$ century -- when AI was firstly recognised as discipline by itself -- several approaches towards machine intelligence became subject of intensive research efforts, leading to the vast corpus of literature and to the abundance of techniques available today.

Notably, two main families of approaches has initially emerged in AI, namely, the \emph{symbolic} and \emph{connectionist} ones \cite{Smolensky1987, SUN2001783}.
%
While the former focuses on representing the world through symbols -- in turn representing concepts --, thus emulating how the human \emph{mind} reasons and infers, the latter aims at mimicking human intuition by emulating how the human \emph{brain} works at a very low level.
%
Despite both families have both pros and cons, they have stepped through both glory and misery---in terms of expectations, funding, research interest, and industry adoption \cite{Hendler2008, russell2016artificial}.

For instance, despite the initial hype, artificial neural networks (NN) -- the warhorse of connectionism -- encountered their first \emph{winter} when Rosenblatt's \emph{perceptron} \cite{rosenblatt1957perceptron} was proven unable to learn the XOR function \cite{Minsky1988}.
%
The period following the publication of the well-known Back-Propagation algorithm \cite{Bryson1979} and the proof that multi-layered perceptrons could be used as universal functional approximators \cite{Cybenko1989}, can be considered as the second \emph{spring} of connectionist approaches.
%
However, at the time -- likely, because of computational limits of the hardware and the lack of data -- the success of connectionist approaches can be considered very moderate, especially when compared with the explosion of deep NN and deep learning (DL) \cite{goodfellow2016deep} which was pervasive in both the academy and the industry during the 2010s, and can thus be considered the third spring of AI.

Even if it is currently not as popular as NN, the history of symbolic AI is extremely important as well---mostly because of the prominent influence it has on the many fields converging in AI.
%
Despite their original ambition of reproducing human reasoning \emph{in toto} has been was inevitably rejected by facts, symbolic approaches gave birth to several research lines which are nowadays considered autonomous fields, such as computational logic \cite{lloyd1990computational}, logic programming \cite{apt1990logic}, planning \cite[Chap. 10-11]{russell2016artificial}, multi agent systems \cite{ferber1999multi}, etc. 

A few decades later, many things has changed.
%
ML, DL, Data Mining \cite{hand2006data}, and Bayesian Inference have enormously widened the spectrum of tasks AI can handle, other than the amount of use cases where AI can be applied.
%
Nevertheless, a dualism is still there, alive and healthy, dividing symbolic approaches from what are now called \emph{sub-symbolic}  ones.

Nowadays, sub-symbolic techniques include NN, but they are not limited to the connectionist techniques.
%
The panorama of sub-symbolic techniques has been widened by the development of several data-mining algorithms -- along with their efficient implementations --, such as SVM \cite{Smola2004}, K-Means, Expectation Maximisation \cite{Dempster77maximumlikelihood}, Viterbi \cite{Viterbi06}, etc, which mostly leverage on \emph{numerical} computations while not being backed by a biological metaphor.

Summarising, at the gates of 2020s, AI consists of a number of powerful techniques -- often backed by sound theoretical or empirical backgrounds --, which are widely employed to automate disparate tasks, both in the industry and in the research.
%
Such tasks, and, in particular, the techniques supporting them, can be categorised within two mostly disjoint families---namely the symbolic and sub-symbolic ones.

\subsubsection{Weak vs.\ Strong AI}

A fundamental question in AI concerns the ultimate goal of the discipline itself.
%
Some say AI should (tend to) produce machines which are actually able to think intelligently -- thus adapt to different situations, understand the context their are situated into, learn from the interactions with the environment and with others --, while say it would be sufficient to create machines simply acting \emph{as if} they were thinking intelligently.
%
The former perspective is classically known as \emph{strong} AI, while the latter is known as \emph{weak} AI.
%
Searle's Chinese Room argument \cite{searle1980} clearly explains the difference among the two by means of a practical example, whereas the well known Turing test \cite{Turing1950} provides a practical means to decide whether a machine's AI is actually strong or not.

Even without discussing the many important and subtle philosophical issues arising from such a dualistic view of AI, we note what follows.
%
The original goal of AI was reaching strong AI.
%
This is likely why, initially, so much hype was put in both symbolic and connectionist approaches.
%
But this is also justifying the strong disappointment which led AI towards its first winter.
%
Most researchers soon realised that weak AI was a far more affordable deal, and this is likely why they stopped seeking generality and started focusing on how to improve each single technique, tailoring them to the domains where they could bring more advantage.

A few years later, the global effect is that a plethora of techniques is available to effectively and efficiently tackle as many tasks.
%
But the glue keeping everything together is still human intelligence \cite{Yao2018}.
%
Indeed, symbolic techniques still require human beings to \emph{handcraft} most complex rules or to \emph{manually} build large knowledge bases.
%
Similarly, most ML-powered models still rely on data scientists to lead their training process.
%
Data scientists are still needed, for instance, to clean-up and pre-process data, and to set up predictors hyper-parameters through their experience, or to leverage on their intuition to interpret how a trained predictor is functioning.

Summarising, the success of AI nowadays is also due to its reduced expectations with respect to what can be delegated to machines.
%
As a side effect, poor care is dedicated in studying how AI could be used to \emph{automate} the many processes involving several, interrelated AI-powered tasks.

\subsubsection{Symbolic vs.\ Sub-symbolic AI}

In the recent years, also because of the rise of XAI, the religious war among the two souls of AI, has hopefully come to an end.
%
More and more authors nowadays are progressively realising and accepting that symbolic and sub-symbolic approaches are \emph{complementary} -- rather than in a competition --, as their combination is now being considered a fruitful way to mutually soften their corners \cite{Hoehndorf2017, xailp-woa2019, lpaas-bdcc2}.
%
In fact, while symbolic approaches are well suited for relatively small-sized problems where complex and exact tasks has to be performed, possibly relying on structured data -- like for instance planning a sequence of actions, finding a path in a graph taking several constrains into an account, deducing information from a prior knowledge base, or learning mathematical relations from vary small data sets --, sub-symbolic approaches are best suited for those use cases where big (up to huge) amounts of possibly unstructured data  must be processed, where errors or lack of precision is tolerated to some extent, if unavoidable---like for instance classifying images or texts, profiling customers by looking at their shopping history, forecasting the weather for a particular area, etc.

As widely recognised by the XAI community, a key weakness of sub-symbolic AI is that it often produces or relies on black-box models which are often hard to (fully) understand for humans \cite{GuidottiMRTGP19}.
%
Such hardness is likely due to the fact that most models and algorithms employed in ML and Data Mining, do not rely on (nor produce) symbolic representations for the knowledge they extract from data, but they rather scatter such knowledge among several elementary units -- like, for instance, neurons in NN, or support vectors in SVM -- which carry very few information if considered one by one, and require a strong cognitive effort for being understood as a whole.
%
Such issues, are not affecting symbolic techniques---especially when symbols are wisely chosen in order to steer humans' intuition.
%
This is because symbols are far closer to what our conscious, rational mind used to handle.
%
For all such reasons, many researchers in the story of AI have argued that a comprehensive approach unifying the two worlds would bring great advantage.

\section{Computational Logic}
\mypapers[section]{cco-softwarex-2021-2pkt,Korner2020HistoryFuturePrologTPLP}

\emph{Computational logic} (CL) \cite{lloyd1990computational} is a fundamental research area for \emph{artificial intelligence} (AI), dealing with formal logic as a means for computing \cite{Paulson2018}.
%
Its penetration into \emph{symbolic AI} is nearly pervasive nowadays, and increasingly going deeper within \emph{sub-symbolic AI} \cite{xaisurvey-ia14,lptech4mas-jaamas35}: CL has enabled the development of the former in the past, and it is now pushing the latter towards interpretability and explainability.
%
Be it exploited to manipulate symbols, or to make sub-symbolic solutions human-intelligible, the common expectation behind CL is to endow software systems with \emph{automated} reasoning.

Generally speaking, automated reasoning involves three major aspects:
%
\begin{inlinelist}
    \item logic,
    \item inference rule, and
    \item resolution strategy.
\end{inlinelist}
%
\emph{Logics} formally define how knowledge is represented and how novel knowledge can be derived from prior one.
%
Each logic comes with several \emph{inference rules}, dictating how to produce new knowledge under particular circumstances.
%
When coupled with some \emph{resolution strategy}, inference rules become deterministic algorithms that computers can execute.

Many logics exist in CL -- e.g. propositional, first-order (FOL), temporal, deontic, etc. --, each one targeting a specific domain.
%
For instance, temporal logic enables reasoning about the chronological ordering of events, deontic logic supports reasoning about permissions/prohibitions and their circumstances, while FOL is general-purpose.
%
Furthermore, different inference rules exist for different logics.
%
Some are \emph{deductive} -- drawing conclusions out of premises --, some are \emph{inductive} -- looking for general rules out of several premises-conclusion examples --, while other are \emph{abductive}---speculating on which premises caused some conclusions.
%
Finally, when a resolution strategy exists for some rule and logic, %-- possibly leveraging some symbolic or sub-symbolic AI technique -- 
it can be reified in software, and used to build intelligent systems capable of automated reasoning.
%
Software of that sort are commonly referred to as a part of the \emph{logic programming} (LP) paradigm~\cite{Nerode1996}.

In LP, programs are typically \emph{theories} (a.k.a.\ \emph{knowledge bases}, KB), i.e.\ collections of sentences in logical form, expressing \emph{facts} and \emph{rules} about some domain, typically in the form of \emph{clauses}, i.e.:
\[ \mathit{Head} \impliedBy \mathit{Body_1},\ ...,\ \mathit{Body_n} \]%,
where both $\mathit{Head}$ and $\mathit{Body_i}$ are \emph{atomic} formul\ae{}, and the whole sentence is read declaratively as logical implication (right-to-left).
%
If $n = 0$, the clause is called \emph{fact}, \emph{rule} otherwiese.
%
An atomic formula is an expression in the form $P(t_1,...,t_m)$ where $P$ is a $m$-ary predicate ($m \geq 0$), and $t_j$ are \emph{terms}.
%
Terms are the most general sort of data structure in LP languages.
%
They can be \emph{constant} (either \emph{numbers} or \emph{atoms}/strings), \emph{variables}, or recursive elements named \emph{structures}.
%
Structures are used to represent clauses, lists, sets, or other sorts of expressions.

Logic \emph{solvers} exploit KB to answer users' \emph{queries} via some inference procedure and resolution strategy.
%
For instance, Prolog interpreters exploit a \emph{deductive} procedure rooted into the SLDNF resolution principle \cite{Kowalski1974, Clark77}, and a depth-first strategy.
%
Yet, other options exist like, e.g., abductive \cite{FungIff97}, inductive \cite{Muggleton94}, probabilistic \cite{RaedtK15} inference.
%
Each of them represents a particular reification of a logic solver.

A common mechanisms in LP is the \emph{unification} algorithm \cite{MartelliMontanari1982} for constructing a \emph{most general unifier} (MGU) for any two suitable terms.
%
Provided that an MGU exists, its subsequent \emph{application} to the terms, makes them syntactically equal.
%
This is a basic brick in virtually all LP algorithms, regardless of the particular inference rule.

Summarising, LP leverages several mechanisms -- terms and clauses representation, knowledge base storage, unification, resolution, etc. --, which constitute the basis of any logic solver.
%
Subsets of these mechanisms may be useful \textit{per se}.

\section{Data Science and Machine Learning}
%
\mypapers[section]{xailp-woa2019,xaisurvey-ia14,agentbasedxai-extraamas2020}

Since several practical AI problems -- such as image recognition, financial and medical decision support systems -- can be reduced to \emph{supervised} ML -- which can be further grouped in terms of either  \emph{classification} or \emph{regression} problems \cite{twala2010,smlreview-faia160} --, in the reminder of this paper we focus on this set of ML problems. 

In those cases, a \emph{learning algorithm} is commonly exploited to estimate the specific nature and shape of an unknown \emph{prediction} function (or \emph{predictor}) $p^*: \mathcal{X} \rightarrow \mathcal{Y}$, mapping each input vector $\mathbf{x}$ from a given input space $\mathcal{X}$ into a prediction from a given output space $\mathcal{Y}$.
%
To do so, the learning algorithm takes into account a number $N$ of examples in the form $(\mathbf{x}_i,\mathbf{y}_i)$ such that $\mathbf{x}_i \in X \subset \mathcal{X}$, $\mathbf{y}_i \in Y \subset \mathcal{Y}$, and $|X| \equiv |Y| \equiv N$.
%
There, each $\mathbf{x}_i$ represents an instance of the input data for which the expected output value $\mathbf{y}_i$ is known or has already been estimated.
%
Such sorts of ML problems are said to be ``supervised'' because the expected targets $Y$ are available, whereas they are said to be ``regression'' problems if $Y$ consists of continuous or numerable values, or ``classification'' problems if $Y$ consists of categorical values. 

The learning algorithm usually assumes $p^* \in \mathcal{P}$, for a given \emph{family} $\mathcal{P}$ of predictors---meaning that the unknown prediction function exists, and it is from $\mathcal{P}$.
%
The algorithm then \emph{trains} a predictor $\hat{p} \in \mathcal{P}$ such that the value of a given loss function $\lambda : \mathcal{Y} \times \mathcal{Y} \rightarrow \mathbb{R}$ -- computing the discrepancy among predicted and expected outputs -- is minimal or reasonably low---i.e.: $\hat{p} = \underset{p \in \mathcal{P}}{\text{argmin}} \left\{ \sum_{i=1}^{N} \lambda(\mathbf{y}_i,\ p(\mathbf{x}_i)) \right\}$.

Depending on the predictor family $\mathcal{P}$ of choice, the nature of the learning algorithm and the admissible shapes of $\hat{p}$ may vary dramatically, as well as the their \emph{interpretability}.
%
Even if the interpretability of predictor families is not a well-defined feature, most authors agree on the fact that some predictor families are \emph{more interpretable than others} \cite{Lipton18}---in the sense that it is easier for humans to understand the functioning and the predictions of the former ones.
%
For instance, it is widely acknowledged that \emph{generalized linear models} (GLM) are more interpretable than neural networks (NN), whereas \emph{decision trees} (DT) \cite{breiman1984classification} are among the most interpretable families \cite{GuidottiMRTGP19}.

\chapter{Representing Data and Knowledge}

\section{Distributed vs. Symbolic}

\section{Intensional vs. Extensional}

\section{Relational vs. Functional}

\chapter{Learning Knowledge from Data}

\section{Machine Learning}

\section{Logic Induction}

\chapter{Generating Data by Reasoning over Knowledge}

\section{Symbolic Reasoning}

\subsection{Inference}

Deduction, Abduction, Induction, Probabilistic, etc.

\subsection{Logic Programming}

\cite{logictech-information11}
\cite{lptech4mas-aamas2021}
\cite{lptech4mas-jaamas35}
\cite{Korner2020HistoryFuturePrologTPLP}

\section{Sub-symbolic Reasoning}

\subsection{Neuro-Symbolic Computation}

\subsection{Knowledge Graph Embedding}

\chapter{Explaining AI via Symbolic Knowledge}

%%% !!!!! ATTENTION !!!!!
%%% this subsubsection is taken from a published paper

Despite the open philosophical issues, it is undeniable that AI and ML are nowadays becoming more and more intertwined with a growing number of aspects of people's every day life \cite{helbing2019, elliott2019}. 
%
In fact, more and more decisions are delegated by humans to software agents whose intelligent behaviour is not the result of some skilled developer endowing it with some clever code, but rather the consequence the agents' capability of learning, planning, or inferring what to do from data.

In spite of the large adoption, intelligent machines whose behaviour is the result of automatic synthesis / learning procedures are difficult to trust for most people---in particular when they are not expert in the field.
%
This is especially true for agents leveraging on machine or deep learning based techniques, often producing models whose internal behaviour is opaque and hard to explain for their developers too.

There, agents often tend to accumulate their knowledge into \emph{black-box} predictive models which are trained through ML or DL.
%
Broadly speaking, the ``black-box'' expression is used to refer to models where knowledge is not explicitly represented -- such as in NN, support vector machines (SVM), or random forests --, and it is therefore difficult, for humans, to understand what a black-box actually knows, or what leads to a particular decision.

Such difficulty in understanding black-boxes content and functioning is what prevents people from fully trusting -- and thus accepting -- them.
%
In several contexts, such as the medical or financial ones, it is not sufficient for intelligent agents to output bare decisions, since, for instance, ethical and legal issues may arise. 
%
An \emph{explanation} for each decision is therefore often desirable, preferable, or even required.  
%
Furthermore, it may happen for instance that black-boxes \emph{silently} learn something wrong (e.g., Google image recognition software that classified black people as gorillas \cite{fourcade2017, crawford2016artificial}), or something right, but in a biased way (like  the ``background bias'' problem, causing for instance husky images to be recognised only because of their snowy background \cite{RibeiroSG16}).

To tackle such trust issues, the \emph{eXplainable Artificial Intelligence} (XAI) research field has recently emerged, and a comprehensive research road map has been proposed by DARPA \cite{darpa2016-xai}, targeting the themes of explainability and interpretability in AI -- and in particular ML -- as a challenge of paramount importance in a world where AI is becoming more and more pervasively adopted.
%
There, DARPA reviews the main approaches to make AI either more interpretable or \emph{a posteriori} explainable, it categorises the many currently available techniques aimed at building meaningful interpretations or explanations for black-box models, it summarises the open problems and challenges, and it provides a reference framework for the researchers interested in the field.

Broadly speaking, research efforts in the field of XAI are focused on achieving key properties in AI, such as \emph{interpretability}, \emph{transparency}, \emph{explainability}, \emph{accountability}, and \emph{trustworthiness}.
%
Unfortunately, such goals are still far from being reached.
%
For instance, as pointed out in \cite{Lipton18}, the aforementioned properties are still lacking a formal and agreed-upon definition.
%
Some authors \cite{Rudin2019} propose to strengthen the adoption of most \emph{interpretable} (i.e. algorithmically transparent) predictive models -- such as generalised linear models or decision trees --, while others seek new ways to produce \emph{post-hoc} explanations capable of tackling even most opaque predictors, such as NN.
%
However, as demonstrated by the comprehensive survey produced by Guidotti et al. \cite{GuidottiMRTGP19}, most works only target classification problems, and they rarely take wider properties -- such as accountability and trustworthiness -- into account.

\sidenote{Spostare il capoverso sotto in un posto più opportuno}
Several works, among the various springs of AI, proposed to extract symbolic knowledge from sub-symbolic models.
%
As witnessed by a number of surveys \cite{GuidottiMRTGP19, GarcezBRFHIKLMS15, AndrewsDT95, GarcezBG01} and works on the topic \cite{BolognaH18, BolognaH16, FrosstH17, JohanssonN09, KrishnanSB1999, HruschkaE2006, ZhouZYS1983, CravenS95, AugastaK12, SatoT2002, KahramanliA09a} -- some of which are from the 80s or the 90s -- the potential of symbolic knowledge \emph{extraction} is well understood, despite not being subject to hype.
%
Unfortunately, a comprehensive and general framework tackling such problem in a general way is still missing.

\note{Armonizzare quanto segue con ciò che precede}

Since the adoption of interpretable predictors usually comes at cost of a lower potential in terms of predictive performance, \emph{explanations} are the newly preferred way for providing understandable predictions without necessarily sacrificing accuracy.
%
The idea, and the main goal of XAI is to create intelligible and understandable explanations for uninterpretable predictors \emph{without} replacing or modifying them.
%
Thus explanations are built through a number of heterogeneous techniques, broadly referred to as \emph{explanators} \cite{GuidottiMRTGP19}---just to cite some, \emph{decision rules} \cite{Augasta2012}, \emph{feature importance} \cite{tolomei2017interpretable}, saliency masks \cite{FongV17}, sensitivity analysis \cite{SundararajanTY17}, etc.

The state of the art for explainability currently recognises two main sorts of explanators, namely, either local or global. 
%
While \emph{local} explanators attempt to provide an explanation for each particular prediction of a given predictor $p$, the \emph{global} ones attempt to provide an explanation for the predictor $p$ as a whole.
%
In other words, local explanators provide an answer to the question ``why does $p$ predict $\mathbf{y}$ for the input $\mathbf{x}$?'' -- such as the LIME technique presented in \cite{RibeiroSG16} --, whereas global explanators provide an answer to the question ``how does $p$ build its predictions?''---such as decision rules.

In spite of the many approaches proposed to explain black boxes, some important scientific questions still remain unanswered. 
%
One of the most important open problems is that, until now, there is no agreement on what an explanation is. 
%
Indeed, some approaches adopt as explanation a set of rules, others a decision tree, others rely on visualisation techniques \cite{GuidottiMRTGP19}. 
%
Moreover, recent works highlight the importance for an explanation to guarantee some properties, e.g., soundness, completeness, and compactness \cite{GuidottiMRTGP19}. 

\note{Probabilmente si può tagliare da qui in poi}

This is why our proposal aims at integrating sub-symbolic approaches with symbolic ones.
%
To this end, DT can be exploited as an effective bridge between the symbolic and sub-symbolic realms.
%
In fact, DT can be easily \emph{(i)} built from an existing sub-symbolic predictor, and \emph{(ii)} translated into symbolic knowledge -- as it is shown in the reminder of this paper -- thanks to their rule-based nature.

Decision trees are an interpretable family of predictors that have been proposed as a \emph{global} means for explaining other, less interpretable, sorts of black-box predictors \cite{TrattnerPR2019,BastaniKB17}---such as neural networks \cite{CravenS95}.
%
The main idea behind such an approach is to build a DT approximating the behaviour of a given predictor, possibly, by only considering its inputs and its outputs.
%
Such approximation essentially trades off predictive performance with interpretability.
%
In fact, the structure of such a DT would then be used to provide useful insights concerning the original predictor inner functioning. 

Describing the particular means for extracting DT from black-boxes is outside the scope of this paper.
%
Given the vast literature on the topic -- e.g., consider reading \cite{GuidottiMRTGP19,AndrewsDT95} for an overview or \cite{CravenS95,JohanssonN09,FrosstH17} for a practical examples -- we simply assume an extracted DT is available and it has an high \emph{fidelity}---meaning that the loss in terms of predictive performance is low, w.r.t. the original black-box.
%
In fact, whereas there exist several works focussing on how to synthesise DT out of black-box predictors, no attention is paid to merging them with symbolic approaches, which can play a key role in enhancing the interpretability and explainability of the system.
%
In this paper we focus on such a matter.

We believe that a logical representation of DT may be interesting and enabling for further research directions.
%
For instance, as far as explainability is concerned, we show how logic-translated DT can be used to both navigate the knowledge stored within the corresponding predictors -- thus acting as \emph{global} explanators --, and produce \emph{narrative} explanations for their predictions---thus acting as \emph{local} explanators. 
%
Note that the restriction on the DT representation makes it easy to map DT onto logical clauses, since DT are finite and with a limited expressivity (if / else conditions).

\section{Explanation vs. Interpretation: Overview}

\note{Essentially report the whole \cite{agentbasedxai-aamas2020} paper}

\section{Symbolic Knowledge Extraction}

\cite{shallow2deep-extraamas2021}
\cite{xailp-woa2019}

\section{Symbolic Knowledge Injection}

\cite{nsc4xai-woa2020}

\part{How}
\label{part:how}

\chapter{The Role of Software}

\chapter{Technological State of the Art}

\cite{coordination-jlamp2020}

\section{Current State of Logic-Based Technologies}

\cite{lptech4mas-aamas2021}
\cite{lptech4mas-jaamas35}
\cite{logictech-information11}

\section{Current State of Machine Learning Technologies}

\section{Current State of XAI Technologies}

\cite{xaisurvey-ia14}

\chapter{Need for An Open Ecosystem for Logic-Based AI}

\cite{cco-softwarex-2021-2pkt}

\chapter{The 2P-Kt Ecosystem}

\cite{cco-softwarex-2021-2pkt}
\cite{kotlindsi4prolog-woa2020}

\chapter{Bridging Logic Programming and Data Processing}

\cite{2pkt-jelia2021}

\chapter{Bridging Logic Programming and Object Orientation}

\cite{cco-softwarex-2021-2pkt}
\cite{kotlindsi4prolog-woa2020}

\chapter{Bridging Logic Programming and Machine Learning}

Castiglio

\chapter{Bridging Logic Programming and XAI}

Psyke

\chapter{Enriching the Ecosystem}

\section{Probabilistic Logic Programming}

Jason

\section{Argumentation}

Pisano

\section{Inductive Logic Programming}

Speciale

\part{Who}
\label{part:who}

\cite{imagination-extraamas2021}
\cite{expectation-extraamas2021}

\chapter{Adding Control to Data via Agents}

\chapter{On the role of Interaction}

\cite{tusow-icccn2019}
\cite{respect-idc2017}
\cite{respectx-comsis15}

\chapter{Blockchain as the way to Trustworthiness}

\cite{bctcoord-bct4mas2018wi}
\cite{bctcoord-bct4mas2019}
\cite{bctcoordination-information11}
\cite{blockchain-goodtechs2018}
\cite{proactivesc-blockchain2019}
\cite{blockchainmas-applsci10}


%----------------------------------------------------------------------------------------
% BIBLIOGRAPHY
%----------------------------------------------------------------------------------------

\backmatter

\part*{}

% \nocite{*} % uncomment this to show all the reference in the .bib file
\bibliographystyle{alpha}
\bibliography{phd-thesis}


\end{document}