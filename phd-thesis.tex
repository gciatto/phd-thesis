\documentclass[12pt,a4paper,openright,twoside]{book}
\usepackage[utf8]{inputenc}
\usepackage{phd-thesis}
\usepackage{code-lstlistings}
\usepackage{notes}
\usepackage{shortcuts}

\school{ALMA MATER STUDIORUM -- UNIVERSITÀ DI BOLOGNA}
\programme{Dottorato di Ricerca in Data Science and Computation}
\title{On the role of Computational Logics in modern Data Science: representing, learning, reasoning, and explaining knowledge}
\author{Giovanni Ciatto}
\date{\today}
\contestsector{09/H1 -- Sistemi di Elaborazione delle Informazioni}
\scientificsector{ING-INF/05 -- Sistemi di Elaborazione delle Informazioni}
\coordinator{Andrea Cavalli}
\supervisor{Andrea Omicini}
\cycle{XXXIII}
\examyear{2022}

\mainlinespacing{1.241} % line spacing in mainmatter, comment to default (1)

\begin{document}
	
\frontmatter
\frontispiece

\begin{abstract}	
Max 2000 characters, strict.
\end{abstract}

\begin{dedication} % this is optional
Optional. Max a few lines.
\end{dedication}

\begin{acknowledgements} % this is optional
Optional. Max 1 page.
\end{acknowledgements}

%----------------------------------------------------------------------------------------
\tableofcontents   
\listoffigures     % (optional) comment if empty
\lstlistoflistings % (optional) comment if empty
%----------------------------------------------------------------------------------------

\mainmatter

%----------------------------------------------------------------------------------------
\chapter{Introduction}
\label{chap:introduction}
%----------------------------------------------------------------------------------------

In the last decade we have witnessed an explosion in the exploitation of artificial intelligence (AI) both in the academy and in the industry, and in virtually all strategical sectors of human expertise.
%
This is not the first time in history that AI attains unprecedented levels of attention, expectation, and fundings, yet it is the first time that such momentum is driven by a pervasive adoption of data science (DS) and, in particular, machine learning (ML).

Nowadays, the tree terms -- AI, DS, and ML -- are often used mistakenly interchangeably, especially by practitioners.
%
Should we speculate on what the causes of such phenomenon are, we would argue this is likely due to the strong hype characterising modern data-driven solutions---both in theory and in practice.
%
This leads both researchers and practitioners to focus on the development of \emph{ML-oriented} frameworks or technologies which, in turn, create a sampling bias making people think that ML exhausts DS, and DS saturates AI.
%
As we further discuss in the subsequent chapters, this is really far from the truth.
%
There are many interesting aspects of AI which lay outside the realm of DS.
%
Notably, in this thesis we focus on computational logics (CL) -- a prominent aspect of AI populating the portion which is not covered by DS -- and its potential role in complementing DS.

As sub-fields of AI, both DS and CL share the common goal of mimicking human intelligence.
%
Of course, they do so in different ways.
%
They focus on different notions and aspects of intelligence, they pursue intelligence through different ways, and for different purposes.
%
Notably, most differences lay in the way CL and DS treat \emph{knowledge}, and, in particular, in the way knowledge is represented, acquired, manipulated, and transferred.

CL, for instance, focuses on \emph{rational} intelligence, and it aims at endowing machines with human-like, automated \emph{reasoning} capabilities.
%
Following this purpose, it relies on \emph{symbolically} represented knowledge, either acquired via logic induction or via manual handcrafting, manipulated via logic inference (e.g. deduction or abduction), and transferred by simply presenting symbols into shared formats.
%
Dually, DS focuses on \emph{intuitive} intelligence, and it aims at endowing humans with statistical tools for mining significant and predictive information from data, in a principled way.
%
When applied to machines, DS provides them with powerful pattern matching, recognition, or stimulus-response capabilities. 
%
For this reason, it relies on sub-symbolically (e.g. \emph{numerically}) represented knowledge, commonly acquired from data via ML, manipulated via algebraic or differential operations, and transferred in disparate, purpose-specific ways.

Of course, both CL and DS come with shortcomings.
%
On the one side, CL commonly requires
%
\begin{inlinelist}
    \item some symbolic knowledge to be eventually handcrafted by humans, manually; and
    \item the task at hand to have a clear formulation, which can be expressed via crisp symbols.
\end{inlinelist}
%
The former issue, clearly hinders scalability, making CL fall short on the knowledge provisioning side.
%
Vice versa, DS is very well suited on this side, as it naturally leverages on scalable algorithms which have been designed mine information semi-automatically from data, possibly scaling up to very large datasets.
%
The latter issue, in turn, makes CL poorly suited to handle fuzzy tasks which are hard to formalise or encode symbolically---think, for instance, to the task of handwritten digits recognition.
%
On the other side, to be effective, DS commonly requires
%
\begin{inlinelist}
    \item very large amounts of data; and
    \item users to be willing and capable of interpreting the numeric results it outputs.
\end{inlinelist}
%
The former issue actually constrains the exploitation of DS into use cases where data is already available or a provisioning procedure is admissible.
%
Vice versa, CL is data efficient as it can bring valuable results even in presence of very small prior knowledge.
%
The interpretability issue is, in turn, among the most relevant topic nowadays.
%
Given the wide exploitation of DS in some many areas of expertise, clarity and intelligibility of its outcomes are becoming a critical aspects---mostly because of their sub-symbolic nature.
%
Vice versa, CL is inherently symbolic in nature and therefore less subject to such interpretability issues.

Accordingly, this thesis stems from the acknowledgement that CS and DS are complementary -- rather than competing -- aspects of AI, and that \emph{knowledge} plays a pivotal role in both these fields.
%
Along this line, we aim to \emph{elicit} and \emph{enable} the many possible bridges among them, w.r.t. knowledge manipualtion.
%
In doing so, we follow the ultimate purpose of increasing the degree of intelligence and autonomy of modern computational systems.
%
Therefore, our focus is on computational entities and on the ways they can combine and integrate CL and DS to either act more intelligently or more autonomosly---or both. 

On the one side, we \emph{elicit} analogies, dichotomies, and possible synergies among CL and DS by analysing them along four orthogonal dimensions, corresponding to as many knowledge-related activities, namely:
%
\begin{description}
    \item[representation] | i.e. the way knowledge is expressed and made interpretable by either machines or human beings, or both; e.g. via symbols, formul\ae, or tensors of real numbers 

    \item[acquisition] | i.e. the way novel knowledge is learned from prior information, mined from data, or attained from external sources; e.g. via data mining, via induction, or via interaction 
    
    \item[inference] |  i.e. the way decisions, suggestions, recommendations, or predictions can be automatically computed out of prior knowledge; e.g. via automated deduction/abduction, or via classification/regression
    
    \item[explanation] | i.e. the way knowledge can be transferred to another entity---be it computational or human
\end{description}

On the other side, we acknowledge that both CL and DS have a prominent overlap with computer science (CS) and software engineering (SE).
%
Regardless of how they manipulate knowledge, both approaches subtend a mathematical modelling of many computational aspects, which must then be reified into well-engineered software technologies to let practitioners actually exploit them. 
%
Accordingly, we further analyse CL and DS from both a computational and technological perspective.
%
While the computational perspective focuses on \emph{what} data structures, algorithms, and workflows they leverage upon to attain intelligence, the technological perspective focuses on \emph{how} such aspects can be translated in practice, via robust software architectures and effective implementations.
%
Along this line, in particular, we assess the current state of the art for technologies laying at the intersection among DS and CL -- or supporting the construction of bridges among the two fields --, identifying holes and proposing lacks to overcome them.
%
The latter in particular is the contribution by which we \emph{enable} the actual combination of CL and DS in practice.

We carry out the whole discussion under an agent-oriented mindset.
%
Within the scope of this document, we call ``agent'' any autonomous entity having its own \emph{locus of control}---be it a human being or a running process programmed software.
%
We may refer to agents as ``intelligent'' in case the come equipped with human-like knowledge-related capabilities, such as the aforementioned capabilities of representing, learning, inferring, or explaining knowledge---or, possibly, a multitude of them.
%
Under such a mindset, human agents are intelligent by definition, whereas software agents may tend to intelligence by acquiring one or more of these capabilities via either CS or DS---or, hopefully, a combination of them.
%
Accordingly, what we have so far called ``machines'' are indeed ``software agents'', and the overall role of the agent-oriented mindset is about focussing on \emph{who} is charge of manipulating knowledge and \emph{when}, other than \emph{where} should knowledge be located in the meanwhile.


\paragraph{Structure of the thesis}

Summarising, the whole thesis discusses in what ways CL and DS can jointly contribute to the management of knowledge within the scope of modern and future intelligent systems, and how technically sound software technologies can be realised along the path.
%
An agent-oriented mindset permeates the whole discussion, by stressing pivotal role of autonomous agents in exploiting both means to reach higher degrees of intelligence.
%
Accordingly, this thesis is organised in three parts, named \emph{What}, \emph{How}, and \emph{Who}, respectively.

In the first part (\emph{What}), we focus on the computational perspective.
%
There, we present the two main branches of AI---namely, the \emph{symbolic} and \emph{sub-symbolic} ones.
%
In particular, we recall classical definitions and survey the current state of the art for both CL and DS.
%
We finally present their strengths and weaknesses, and the many possible bridges among them, w.r.t. knowledge representation, acquisition, manipulation, and transfer.

In the second part (\emph{How}), we focus on the technological perspective.
%
There, we survey the current state of technologies.
%
In particular, we focus on logic-based technologies laying at the intersection with DS, identifying the current lacks w.r.t. current theoretical advances. 
%
We then propose the notion of AI ecosystem, and \twopkt{} as its main reification in software.
%
Finally, we present a number of possible ways to extend the \twopkt{} ecosystem towards DS.

Finally, the third part (\emph{Who}) closes the loop by discussing the role of agents \ldots
\note{TBD}


\part{What}
\label{part:what}

\chapter{Historical Perspective on AI}

AI is a multi-faceted discipline leveraging on contributions coming from several areas of human knowledge, there including Mathematics, Computer Science, Statistics, Psychology, Philosophy, and many others.
%
A famous and comprehensive survey on the many aspects of AI is proposed by Russell and Norving in \cite{russell2016artificial}.
%
In the second half the $20^{th}$ century -- when AI was firstly recognised as discipline by itself -- several approaches towards machine intelligence became subject of intensive research efforts, leading to the vast corpus of literature and to the abundance of techniques available today.

Notably, two main families of approaches has initially emerged in AI, namely, the \emph{symbolic} and \emph{connectionist} ones \cite{Smolensky1987, SUN2001783}.
%
While the former focuses on representing the world through symbols -- in turn representing concepts --, thus emulating how the human \emph{mind} reasons and infers, the latter aims at mimicking human intuition by emulating how the human \emph{brain} works at a very low level.
%
Despite both families have both pros and cons, they have stepped through both glory and misery---in terms of expectations, funding, research interest, and industry adoption \cite{Hendler2008, russell2016artificial}.

For instance, despite the initial hype, artificial neural networks (NN) -- the warhorse of connectionism -- encountered their first \emph{winter} when Rosenblatt's \emph{perceptron} \cite{rosenblatt1957perceptron} was proven unable to learn the XOR function \cite{Minsky1988}.
%
The period following the publication of the well-known Back-Propagation algorithm \cite{Bryson1979} and the proof that multi-layered perceptrons could be used as universal functional approximators \cite{Cybenko1989}, can be considered as the second \emph{spring} of connectionist approaches.
%
However, at the time -- likely, because of computational limits of the hardware and the lack of data -- the success of connectionist approaches can be considered very moderate, especially when compared with the explosion of deep NN and deep learning (DL) \cite{goodfellow2016deep} which was pervasive in both the academy and the industry during the 2010s, and can thus be considered the third spring of AI.

Even if it is currently not as popular as NN, the history of symbolic AI is extremely important as well---mostly because of the prominent influence it has on the many fields converging in AI.
%
Despite their original ambition of reproducing human reasoning \emph{in toto} has been was inevitably rejected by facts, symbolic approaches gave birth to several research lines which are nowadays considered autonomous fields, such as computational logic \cite{lloyd1990computational}, logic programming \cite{apt1990logic}, planning \cite[Chap. 10-11]{russell2016artificial}, multi agent systems \cite{ferber1999multi}, etc. 

A few decades later, many things has changed.
%
ML, DL, Data Mining \cite{hand2006data}, and Bayesian Inference have enormously widened the spectrum of tasks AI can handle, other than the amount of use cases where AI can be applied.
%
Nevertheless, a dualism is still there, alive and healthy, dividing symbolic approaches from what are now called \emph{sub-symbolic}  ones.

Nowadays, sub-symbolic techniques include NN, but they are not limited to the connectionist techniques.
%
The panorama of sub-symbolic techniques has been widened by the development of several data-mining algorithms -- along with their efficient implementations --, such as SVM \cite{Smola2004}, K-Means, Expectation Maximisation \cite{Dempster77maximumlikelihood}, Viterbi \cite{Viterbi06}, etc, which mostly leverage on \emph{numerical} computations while not being backed by a biological metaphor.

Summarising, at the gates of 2020s, AI consists of a number of powerful techniques -- often backed by sound theoretical or empirical backgrounds --, which are widely employed to automate disparate tasks, both in the industry and in the research.
%
Such tasks, and, in particular, the techniques supporting them, can be categorised within two mostly disjoint families---namely the symbolic and sub-symbolic ones.

\subsubsection{Weak vs.\ Strong AI}

A fundamental question in AI concerns the ultimate goal of the discipline itself.
%
Some say AI should (tend to) produce machines which are actually able to think intelligently -- thus adapt to different situations, understand the context their are situated into, learn from the interactions with the environment and with others --, while say it would be sufficient to create machines simply acting \emph{as if} they were thinking intelligently.
%
The former perspective is classically known as \emph{strong} AI, while the latter is known as \emph{weak} AI.
%
Searle's Chinese Room argument \cite{searle1980} clearly explains the difference among the two by means of a practical example, whereas the well known Turing test \cite{Turing1950} provides a practical means to decide whether a machine's AI is actually strong or not.

Even without discussing the many important and subtle philosophical issues arising from such a dualistic view of AI, we note what follows.
%
The original goal of AI was reaching strong AI.
%
This is likely why, initially, so much hype was put in both symbolic and connectionist approaches.
%
But this is also justifying the strong disappointment which led AI towards its first winter.
%
Most researchers soon realised that weak AI was a far more affordable deal, and this is likely why they stopped seeking generality and started focusing on how to improve each single technique, tailoring them to the domains where they could bring more advantage.

A few years later, the global effect is that a plethora of techniques is available to effectively and efficiently tackle as many tasks.
%
But the glue keeping everything together is still human intelligence \cite{Yao2018}.
%
Indeed, symbolic techniques still require human beings to \emph{handcraft} most complex rules or to \emph{manually} build large knowledge bases.
%
Similarly, most ML-powered models still rely on data scientists to lead their training process.
%
Data scientists are still needed, for instance, to clean-up and pre-process data, and to set up predictors hyper-parameters through their experience, or to leverage on their intuition to interpret how a trained predictor is functioning.

Summarising, the success of AI nowadays is also due to its reduced expectations with respect to what can be delegated to machines.
%
As a side effect, poor care is dedicated in studying how AI could be used to \emph{automate} the many processes involving several, interrelated AI-powered tasks.

\subsubsection{Symbolic vs.\ Sub-symbolic AI}

In the recent years, the historical dichotomy between the ``two souls'' of AI has been reconciled, in favour of a comprehensive vision where symbolic and sub-symbolic approaches are seen as \emph{complementary} -- rather than in a competition -- so that they mutually soften their corners \cite{Hoehndorf2017, xailp-woa2019, lpaas-bdcc2}.
%
While symbolic approaches are well suited for relatively small-sized problems where complex and exact tasks has to be performed, possibly relying on structured data -- like for instance planning a sequence of actions, finding a path in a graph taking several constrains into an account, deducing information from a prior knowledge base, or learning mathematical relations from vary small data sets --, sub-symbolic approaches are best suited for those use cases where big (up to huge) amounts of possibly unstructured data  must be processed, where errors or lack of precision is tolerated to some extent, if unavoidable---like for instance classifying images or texts, profiling customers by looking at their shopping history, forecasting the weather for a particular area, etc.
%
Such issues, are not affecting symbolic techniques---especially when symbols are wisely chosen in order to steer humans' intuition.
%
This is because symbols are far closer to what our conscious, rational mind used to handle.
%
For all such reasons, many researchers along the history of AI have argued that a comprehensive approach unifying the two worlds would bring great advantage.

More precisely, complementarity between symbolic and sub-symbolic AI naturally emerges when comparing the two approaches under the following perspectives:
%
\begin{itemize}
    \item sub-symbolic AI is \emph{opaque}, meaning that human beings struggle in understanding the functioning and behaviour of sub-symbolically intelligent systems; instead, symbolic AI is more \emph{transparent}, as it is both human- and machine-interpretable at the same time
    %
    \item sub-symbolic AI can improve itself \emph{automatically} by consuming data, but it is difficult to extend and re-use outside the contexts it was designed for; conversely, symbolic AI is flexible and extensible, but requires humans to \emph{manually} provide symbolic knowledge
    %
    \item sub-symbolic AI is adequate for \emph{fuzzy} problems where some (minimal) degree of error or uncertainty can be tolerated; whereas symbolic AI calls for precise data and queries provided by human beings, yet provides exact, \emph{crisp} results as its outcome.
\end{itemize}

In the remainder of this chapter, we provide a brief overview of the two major disciplines laying within the scope of symbolic or sub-symbolic AI, respectively.
%
These are computational logic -- a prominent branch of symbolic AI leveraging on logics to per form any knowledge-related task, ranging from representation to inference --, and data science -- a prominent branch of sub-symbolic AI leveraging on statistics to manipulate data and mine knowledge out of them.

\section{Computational Logic}
\mypapers[section]{cco-softwarex-2021-2pkt,Korner2020HistoryFuturePrologTPLP,logictech-information11}

\emph{Computational logic} (CL) \cite{lloyd1990computational} is a fundamental research area for \emph{artificial intelligence} (AI), dealing with formal logic as a means for computing \cite{Paulson2018}.
%
Its penetration into \emph{symbolic AI} is nearly pervasive nowadays, and increasingly going deeper within \emph{sub-symbolic AI} \cite{xaisurvey-ia14,lptech4mas-jaamas35}: CL has enabled the development of the former in the past, and it is now pushing the latter towards interpretability and explainability.
%
Be it exploited to manipulate symbols, or to make sub-symbolic solutions human-intelligible, the common expectation behind CL is to endow software systems with \emph{automated} reasoning.

Generally speaking, automated reasoning involves three major aspects:
%
\begin{inlinelist}
    \item logics,
    \item inference rules, and
    \item resolution strategies.
\end{inlinelist}

\emph{Logic} formally defines how knowledge is represented and how novel knowledge can be derived from prior one.
%
Each logic comes with several \emph{inference rules}, dictating how to produce new knowledge under particular circumstances.
%
When coupled with some \emph{resolution strategy}, inference rules become deterministic algorithms that computers can execute to reason autonomously.

Many logics exist in CL -- e.g. propositional, first-order (FOL), temporal, deontic, etc. --, each one targeting a specific way of reasoning.
%
For instance, temporal logic enables reasoning about the chronological ordering of events, deontic logic supports reasoning about permissions/prohibitions and their circumstances, while FOL is general-purpose.
%
Furthermore, different sorts of inference rules exist for different logics.
%
Some are \emph{deductive} -- drawing conclusions out of premises --, some are \emph{inductive} -- looking for general rules out of several premises-conclusion examples --, while other are \emph{abductive}---speculating on which premises caused some conclusions.
%
Finally, when a resolution strategy exists for some rule and logic, it can be reified in software, and used to build intelligent systems capable of automated reasoning.
%
Software of that sort are commonly referred to as a part of the \emph{logic programming} (LP) paradigm~\cite{Nerode1996}.

In LP, programs are typically \emph{theories} (a.k.a.\ \emph{knowledge bases}, KB), i.e.\ collections of sentences in logical form, expressing \emph{facts} and \emph{rules} about some domain, typically in the form of \emph{clauses}, i.e. expressions connecting a number of interrelated \emph{predicates} via logic connectives (e.g. operators such as $\wedge$, $\vee$, $\rightarrow$, $\leftrightarrow$, $\lnot$, etc.).
 %, i.e.:
% \[ \mathit{Head} \impliedBy \mathit{Body_1},\ ...,\ \mathit{Body_n} \]%,
% where both $\mathit{Head}$ and $\mathit{Body_i}$ are \emph{atomic} formul\ae{}, and the whole sentence is read declaratively as logical implication (right-to-left).
% %
% If $n = 0$, the clause is called \emph{fact}, \emph{rule} otherwiese.
% %
% An atomic formula is an expression in the form $P(t_1,...,t_m)$ where $P$ is a $m$-ary predicate ($m \geq 0$), and $t_j$ are \emph{terms}.
% %
% Terms are the most general sort of data structure in LP languages.
% %
% They can be \emph{constant} (either \emph{numbers} or \emph{atoms}/strings), \emph{variables}, or recursive elements named \emph{structures}.
% %
% Structures are used to represent clauses, lists, sets, or other sorts of expressions.
%
There, predicates represent statements describing or relating one or more entities about the domain at hand.

Depending on the particular logic of choice, predicates -- and therefore clauses -- may, carry \emph{variables}, i.e. placeholders for unknown entities, and possibly quantifiers for those variables ($\exists$ or $\forall$).
%
Some logics may also endow clauses with further information, such as for instance \emph{probabilities} -- describing the degree of likelihood for a clause to hold true --, or \emph{modalities}---describing the context in which a clause may hold (e.g. \emph{when}).
%
In any case, logic information is represented in such a way that both human and computational agents can interpret and manipulate it, in principle.

One powerful trait of logics is that they enable the representation of complex, intricate, or infinite domains \emph{intensively} (i.e. implicitly) rather than explicitly---e.g. via multiple recursive clauses.
%
So, if a domain involves an infinite amount of entites, these do not necessarily require an infinite amount of memory to be represented.
%
For instance, the set of natural integers can be represented in logic using just two FOL clauses---of which, one is recursive.

Software agents devoted to automated reasoning via LP are commonly referred to as logic \emph{solvers}.
%
They rely on pre-existing KB to derive inferences via some inference procedure and resolution strategy.
%
They may do so either \emph{reactively}, -- i.e. in response to some external stimulus, e.g. some user's \emph{query} --, or \emph{pro-actively}---i.e. spontaneously, in order to reach some goal, e.g. computing the optimal path before moving. 
%
Prolog-based solvers \cite{ColmerauerR93,colmerauer1986-theoreticalProlog}, for instance, exploit a \emph{deductive} procedure rooted into the SLDNF resolution principle \cite{Kowalski1974, Clark77}, and a depth-first strategy.
%
They commonly do so in response to users' queries, provided via a textual interface.
%
Yet, a number of Prolog solvers exists supporting the same inference procedure via different strategies (e.g. tabled-resolution\missingref) or as well as entirely different principles (e.g. Constraint Logic Programming). 
%
Of course, other options exist targetting other logics as well, like, e.g., abductive \cite{FungIff97}, inductive \cite{Muggleton94}, probabilistic \cite{RaedtK15} inference.
%
Each of them represents a particular reification of a logic solver.

\paragraph{Limits of CL.}

Despite the many possibilities, however, there are a number of issues which are not tied to any particular choice of logic, inference procedure, or resolution strategy, but they are rather inherent to CL itself.
%
Such issues involve
%
\begin{inlinelist}
    \item decidability,
    \item tractability,
    \item knowledge acquisition, and
    \item symbols grounding.
\end{inlinelist}

Decidability and tractability deal with the theoretical questions: ``can a logic solver provide an answer to any logic query it receives? can it do so in reasonable time?''.
%
Such aspects are deeply entangled with the particular logic the solver is leveraging upon.
%
Depending on which and how many features a logic includes, it may be more or less \emph{expressive}.
%
The higher the expressiveness, the more the complexity of the problems which may be represented via logic and processed via inference increases.
%
This opens to the possibility, for the solver, to meet queries which cannot be answered in useful time, or relying upon a limited amount of memory, or at all.
%
Roughly speaking, more expressive logic languages make it easier for human beings to describe a particular domain -- usually, requiring them to write less and more concise clauses --, at the expense of an higher difficulty for software agents to draw inferences autonomously---because of computational \emph{tractability}.
%
This is a well-understood phenomenon in both CS and CL \cite{LevesqueB87, BrachmanL2004}, often referred to as the \emph{expressiveness--tractability} trade-off.
%
In practice, however, a good trade-off is represented by FOL and its subsets (e.g. Horn logic\missingref), or modal variants (e.g. linear temporal logic\missingref).
%
Despite consisting of Turing-equivalent formalisms -- for which the existence of undecidable or intractable situations cannot be excluded in the general case --, they come with sufficiently wide representational capabilities and effective inference procedures, making them exploitable in practice---provided that human developers avoid writing undecidable/intractable algorithms.

% A common mechanisms in LP is the \emph{unification} algorithm \cite{MartelliMontanari1982} for constructing a \emph{most general unifier} (MGU) for any two suitable terms.
% %
% Provided that an MGU exists, its subsequent \emph{application} to the terms, makes them syntactically equal.
% %
% This is a basic brick in virtually all LP algorithms, regardless of the particular inference rule.

% Summarising, LP leverages several mechanisms -- terms and clauses representation, knowledge base storage, unification, resolution, etc. --, which constitute the basis of any logic solver.
% %
% Subsets of these mechanisms may be useful \textit{per se}.

Knowledge acquisition deals with the question ``where does the knowledge solvers reason upon come from'', or alternatively: ``who is in charge of constructing knowledge bases''?
%
Recalling that logic clauses may become arbitrarily complex and represent possibly infinite domains in a very concise way, it is unsurprising that the burden of knowledge production is mostly on humans.
%
Unfortunately, this implies the degree of automatism in knowledge production is pretty low, as well as the scalability of the approach.
%
Many attempts have been performed over the years to distil human knowledge into symbolic form to formalise common-sense for software agents.
%
To date, there exist a number of common-sense knowledge bases and ontologies, supporting practical textual-reasoning tasks on real-world documents including analogy-making, and other context oriented inferences---see for instance \missingref[lieberman2004, trinh2018, liu2004conceptnet, liu2002goose, shapiro1999sneps from information-2020]).
%
Yet, most of these solutions are either semi-automatically constructed, or community driven -- when not both --, therefore exhaustiveness, consistency, or coherence may be lacking.
%
There have also been a number of attempts to construct very large knowledge bases of commonsense knowledge by hand, one of the largest being the CYC program by Douglas Lenat at CyCorp \missingref[lenat1995-cyc from information-2020]---which is, however, only usable behind payment.

Finally, symbols grounding deals with the problem of letting software agents provide semantics for the symbols they manipulate.
%
Put it simply, we may tell a logic solver that Abraham is the father of Isaac -- $father(isaac, abraham)$ --, and also that, for all possible $X$ and $Y$, if $X$ is the father of $Y$, then $Y$ must be the child of $X$ -- $father(X, Y) \rightarrow child(Y, X)$ --, and the solver may also be able to infer that Isaac is thus the child of Abraham, while still having no idea of how to recognise Isaac, Abraham, nor the fatherhood relation, were it written in another form. 
%
In other words, the bindings between the symbols processed by software agents and the corresponding entities from the real world are hard to establish and maintain for a bare logic solver---unless other mechanisms are in place.
%
This issue will hardly be solved within the symbolic world alone, as it is deeply entangled with the problem of letting a software agent perceive the external world via sensors, and recognising the objects therein contained.
%
The latter problem is inherently sub-symbolic as it requires acquiring, processing, and fusing raw data coming from the sensors.

\section{Data Science and Machine Learning}
%
\mypapers[section]{xailp-woa2019,xaisurvey-ia14,agentbasedxai-extraamas2020}

Data science (DS) is a relatively young discipline laying at the intersection among AI, Statistics, CS and SE.
%
It essentially deals with the extraction of relevant information out of data, and, in particular, with the \emph{data-driven} creation of \emph{predictive} models of real world phenomena.
%
Thanks to its focus on predictive models, DS is applied to virtually all statistical sciences, ranging from physics to law, stepping through biology, healthcare, or finance.
%
In all such scenarios, the reliance on real-world data is quintessential to tune such predictive models in such a way to make them adhere to reality.

As a science, DS studies the many means one can exploit to % whenever in need to
%
\begin{inlinelist}
    \item let a software agent learn new behaviours from examples, which would otherwise be hard to encode for human developers (e.g. handwritten text recognition),
    \item automatically recognise patterns of similar objects given a number of examples (e.g. face detection),
    \item detect recurrent patterns in data, even in lack of prior examples (e.g. customer profiling),
    \item predict the future evolution of a phenomenon given its historical data (e.g. stock performance predictor),
    \item simulate the dynamics of complex phenomena (e.g. weather forecasting),
    \item figure out the mathematical relation biding two or more variables, from a number of samples (e.g. studying estate market prices),
    \item fuse data coming from different sources to infer unobservable measures (e.g. indoor localization), etc.
\end{inlinelist}
%
other than, of course the theories and practices to assess and increase the predictive performance of all such models.
%
In doing so, DS borrows countless algorithms, methods, and techniques from disparate fields, including but not limited to machine learning (there including supervised, unsupervised, and reinforcement learning), data mining, Bayesian inference, statistics, etc.

Despite the plethora of algorithms and methods which lay nowadays under the DS umbrella, a concise overview of the discipline can be outlined in terms of \emph{tasks}.
%
Several algorithms can be used in DS to perform a well-established pool of data-analytics tasks having a clear knowledge-related goal.
%
Most common tasks in DS are for instance:
%
\begin{description}
    \item[function fitting] (a.k.a. classification or regression) | i.e. the \emph{supervised} learning task of inferring the input-output relation among a number samples, to be later able to estimate likely outputs for novel, unseen inputs;
    \item[clustering] | i.e. the \emph{unsupervised} learning task of finding similar groups of instances in a dataset -- according to a given notion of \emph{similarity} or distance --, to be later able to classify novel instances according to some group;
    \item[anomaly detection] | i.e. the \emph{unsupervised} learning task of tuning an algorithm to discern ``normal'' situations from exceptional ones, provided that some historical data is available, to later be able to detect the latter;
    \item[filtering] (resp. \textbf{smoothing} or \textbf{forecasting}) | i.e. the Bayesian task of estimating the unknown \emph{current} (resp. \emph{past} or \emph{future}) state of a system given a sample of observations capturing the evolution of a number of variables which should depend of that state;  
    \item[most likely explanation] | like the above, but focussing on the most likely \emph{sequence} of states a system has traversed.
\end{description}
%
Most of these tasks may be implemented in several ways and by several algorithms.
%
For instance, function fitting alone may be realised using neural networks, (generalised) linear models, support vector machines, decision trees, and many others.

For all such tasks, two major phases are commonly identified by DS, namely \emph{training} (a.k.a. learning, or fitting) and inference (a.k.a. usage).
%
The first phase (training) commonly occurs behind the scenes, and it is conducted by \emph{data scientists} -- i.e. human being --, despite involving some semi-automatic tasks.
%
The second phase (inference) is commonly what AI consumers use and deal with.
%
During training, the most adequate algorithm is selected for the data and the task at hand, and it is then tranined on data, producing a predictive \emph{model} which, hopefully, is predictive enough to be later used on novel data.
%
Users may then exploit the model by feeding it with novel data, to draw predictions.

Notably, regardless of their technical details, all the data-analytics tasks above may leverage on a number of lower level activities which are orthogonal w.r.t. the choice of the particular implementing algorithm, as they involve routine operations, assessment procedures, or best practices which are commonly executed either before or after the data-analytics task itself.
%
Examples of these kind of activities are, for instance:
%
\begin{description}
    \item[feature engineering] | i.e. a whole class of pre-processing techniques -- such normalising numeric data into predefined intervals, changing the way data is encoded, or creating new data from the available one -- which may be used to improve or transform the available data, in order to improve the performance of the data analytics task;
    \item[dimensionality reduction] | i.e. a whole class of data manipulation techniques aimed at selecting the most relevant attributes of data for a given data-analytics task (before running the task)---such as principal component analysis;
    \item[model assessment] i.e. a whole class of statistical methods, algorithms, and practices to assess the performance of the data-analytic task in a principle way---such as for instance supervised learning metrics (e.g. accuracy, or mean-squared error) or procedures such as cross-validation or test set separation;
    \item[model selection] i.e. a number of strategies, approaches, and practices to let data scientiest select the best predictive model when multiple options fit the data and the situation at hand---e.g. by leveraging on \emph{cross-validation}, possibly in combination with a grid search strategy to identify the most adequate type of pre 
\end{description}
%
Despite being really important for the practice of data science, such aspects are quite marginal within the scope of this thesis.
%
Accordingly, in the following chapters we focus data-analytics tasks, and in particular supervised learning ones. 

\paragraph{}

% Data analytics
%     - general goals: learning otherwise hard-to-formalise tasks, recognise patterns, detect patterns in data, predicting the future knowing the past, creating predictive models from data + plus all collateral activities
%     - tasks: supervised learning, unsurpervised learning, feature engineering, bayesian modelling, reinforcement learning
%     - practices: function fitting, clustering, dimensionality reduction, feature selection, filtering, smoothing, forecasting, data fusion
%     - inference vs. fitting

% Big data processing:
%     - general goals: online processing of large or unlimited streams of data, offline processing of large amounts of data, storage and querying of large amounts of data,




\subsection{Machine Learning}

Since several practical AI problems -- such as image recognition, financial and medical decision support systems -- can be reduced to \emph{supervised} ML -- which can be further grouped in terms of either  \emph{classification} or \emph{regression} problems \cite{twala2010,smlreview-faia160} --, in the reminder of this paper we focus on this set of ML problems. 

In those cases, a \emph{learning algorithm} is commonly exploited to estimate the specific nature and shape of an unknown \emph{prediction} function (or \emph{predictor}) $p^*: \mathcal{X} \rightarrow \mathcal{Y}$, mapping each input vector $\mathbf{x}$ from a given input space $\mathcal{X}$ into a prediction from a given output space $\mathcal{Y}$.
%
To do so, the learning algorithm takes into account a number $N$ of examples in the form $(\mathbf{x}_i,\mathbf{y}_i)$ such that $\mathbf{x}_i \in X \subset \mathcal{X}$, $\mathbf{y}_i \in Y \subset \mathcal{Y}$, and $|X| \equiv |Y| \equiv N$.
%
There, each $\mathbf{x}_i$ represents an instance of the input data for which the expected output value $\mathbf{y}_i$ is known or has already been estimated.
%
Such sorts of ML problems are said to be ``supervised'' because the expected targets $Y$ are available, whereas they are said to be ``regression'' problems if $Y$ consists of continuous or numerable values, or ``classification'' problems if $Y$ consists of categorical values. 

The learning algorithm usually assumes $p^* \in \mathcal{P}$, for a given \emph{family} $\mathcal{P}$ of predictors---meaning that the unknown prediction function exists, and it is from $\mathcal{P}$.
%
The algorithm then \emph{trains} a predictor $\hat{p} \in \mathcal{P}$ such that the value of a given loss function $\lambda : \mathcal{Y} \times \mathcal{Y} \rightarrow \mathbb{R}$ -- computing the discrepancy among predicted and expected outputs -- is minimal or reasonably low---i.e.: $\hat{p} = \underset{p \in \mathcal{P}}{\text{argmin}} \left\{ \sum_{i=1}^{N} \lambda(\mathbf{y}_i,\ p(\mathbf{x}_i)) \right\}$.

Depending on the predictor family $\mathcal{P}$ of choice, the nature of the learning algorithm and the admissible shapes of $\hat{p}$ may vary dramatically, as well as the their \emph{interpretability}.
%
Even if the interpretability of predictor families is not a well-defined feature, most authors agree on the fact that some predictor families are \emph{more interpretable than others} \cite{Lipton18}---in the sense that it is easier for humans to understand the functioning and the predictions of the former ones.
%
For instance, it is widely acknowledged that \emph{generalized linear models} (GLM) are more interpretable than neural networks (NN), whereas \emph{decision trees} (DT) \cite{breiman1984classification} are among the most interpretable families \cite{GuidottiMRTGP19}.

\chapter{Representing Data and Knowledge}

\emph{Knowledge representation} has always been regarded as a key issue since the early days of AI, since no reasoning can exist without knowledge.
%
Its instances spread from \emph{deductive data bases} \cite{green1968} to \emph{description logics} \cite{baader2002} and \emph{ontologies} \cite{cimiano2006-ontologies}, to name just a few. 
%
Many kinds of logic-based knowledge representation systems have been proposed over the years, mostly relying on description logics and modal logics to represent, respectively, terminological knowledge and time-dependent or subjective knowledge.

Early KR formalisms, such as \emph{semantic networks} and \emph{frames} \cite{sowa2014}, also aimed at providing a structured representation of information.
%
Description logics are based on the same ideas, but without the formal weaknesses that made the use of their precursors problematic. 
%
In DL, complex concepts are built from simpler ones with an emphasis on the decidability of the reasoning tasks and on the provision of sound, complete and empirically-tractable reasoning services. 
%
Applications range from reasoning with database schemas and queries \cite{artale2002} to \emph{ontology languages} such as OIL, DAML+OIL and OWL \cite{horrocks2005}---always keeping in mind that not only the key inference problems should be decidable, but also that the decision procedures should be implemented \emph{efficiently}.

Ontology-based approaches are popular because of their basic goal---a common understanding of some domain that can be shared between people and application systems.
%
At the same time, it should be understood that the general concepts and relations of a top-level ontology can rarely accommodate all of the systems peculiarities \cite{van2008,valente2005}.

A number of systems based on DL have been developed -- e.g. \cite{cohen1994,moller2003} -- in diverse application domains, such as natural language processing, configuration of technical systems, software information systems, optimising queries to databases, planning.

\section{Distributed vs. Symbolic}

\section{Intensional vs. Extensional}

\section{Relational vs. Functional}

\chapter{Learning Knowledge from Data}

\section{Machine Learning}

\section{Logic Induction}

\chapter{Generating Data by Reasoning over Knowledge}

\section{Symbolic Reasoning}

Logic-based reasoning approaches root back to John McCarthy's work of 1958 \cite{Mccarthy1958}, aimed at developing the idea of formalising the so-called \emph{commonsense reasoning} to build \emph{intelligent artifacts}. 
%
This also means to consider non-trivial involved issues, such as the need of formalising the situation, actions and causal laws, etc.
%
% Many tools have been developed over the years for commonsense reasoning formalisation: there are freely available commonsense knowledge bases and natural language processing toolkits, supporting practical textual-reasoning tasks on real-world documents including analogy-making, and other context oriented inferences---see for instance \cite{lieberman2004,trinh2018,liu2004conceptnet,liu2002goose,shapiro1999sneps}). 
% %
% There have been also a number of attempts to construct very large knowledge bases of commonsense knowledge by hand, one of the largest being the CYC program by Douglas Lenat at CyCorp \cite{lenat1995-cyc}. 

The modern approach to automated theorem proving starts with Robinson's resolution principle \cite{robinson1965}: since then, several technologies have exploited \emph{deduction} on first-order logic knowledge base to provide reasoning capabilities in diverse areas---logic programming, deductive data bases, and constraint logic programming (CLP) possibly being the major ones.
%
Other approaches and techniques, however, built upon the \emph{induction} and \emph{abduction} principles.

As its name suggests, \emph{deduction} operates top-down, deriving a true conclusion from a universal true premise: logically speaking, this means that the conclusion's truth necessarily follows from the premise's truth.
%
\emph{Induction}, instead, operates bottom-up, basically making a guess -- a generalization -- from specific known facts: so, the reasoning involves an element of probability, as the conclusion is not based on universal premises.
%
\emph{Abduction} is somehow similar, but seeks for cause-effect relationships---i.e., the goal is to find out under which hypotheses (or premises) a certain goal is provable. 
%
Such technologies are exploited, in particular, for the verification of compliance of specific properties \cite{montali2010}. 

\emph{Logic programming} (LP) is likely the most widely-adopted technology based on deduction. 
%
From Colmerauer and Kowalsky's seminal work \cite{Kowalski1974,colmerauer1986-theoreticalProlog}, the Prolog language has been since then one of the most exploited language in AI applications \cite{Dawson1996}. 
%
Other valuable approaches include \emph{fuzzy logic}, \emph{answer-set programming} (ASP), \emph{constraint logic programming} (CLP), \emph{non-monotonic reasoning}, and \emph{belief-desire-intention} (BDI).

Fuzzy logic \cite{yen1999} aims at dealing with lack of precision or uncertainty.
%
In this sense, it is perhaps closer in spirit to the human thinking than traditional logic systems. 
%
Not surprisingly, fuzzy approaches are exploited as a key technology in specific application areas, e.g., the selection of manufacturing technologies \cite{goyal2012}, and industrial processes where the control via conventional methods suffers from the lack of quantitative data about I/O relations.
%
There, a fuzzy logic controller effectively synthesises an automatic control strategy from a linguistic control strategy based on an expert's knowledge. 

\emph{Answer set programming} (ASP) and \emph{constraint logic programming} (CLP) are the two main logical paradigms for dealing with various classes of NP-complete combinatorial problems. 
%
ASP solvers are aimed at computing the answer sets of standard logic programs; these tools can be seen as theorem provers, or model builders, enhanced with several built-in heuristics to guide the exploration of the solution space.
%
% Some of the best known solvers are Clingo \cite{gebser2014-clingo} and DLV \cite{eiter2000}.

Constraint logic programming (CLP) \cite{jaffar1987}, perhaps the most natural extension of LP (or, its most relevant generalisation), has evolved over the years into a powerful programming paradigm, widely used to model and solve hard real-life problems \cite{Rossi2000} in diverse application domains---from circuit verification to scheduling, resource allocation, timetabling, control systems, etc.
%
CLP technologies can be seen as complementary to \emph{operation research} (OR) techniques: while OR is often the only way to find the optimal solution, CLP provides generality, together with a high-level modelling environment, search control, compactness of the problem representation, constraint propagation, and fast methods to achieve a valuable solution \cite{rossi2008}.
%
% CLP tools evolved from the  ancestor -- CHIP \cite{simonis1995-chip}, the first to adopt constraint propagation -- to the constraint-handling libraries of ILOG \cite{ilog} and COSYTEC \cite{aggoun1993}, up to CLP languages such as Prolog III \cite{colmerauer1990}, Prolog IV \cite{benhamou1995}, CLP(R) \cite{jaffar1992}, and clp(fd) \cite{codognet1996}.

Non-monotonic reasoning means to face the basic objection \cite{minsky1975} that logic could not represent knowledge and commonsense reasoning as humans because the human reasoning is inherently \emph{non monotonic}---that is, consequences are not always preserved, in contrast to first-order logic.
%
Since then, a family of approaches have been developed to suit specific needs---among these, \emph{default reasoning} \cite{reiter1980}, \emph{defeasible reasoning} \cite{pollock1987}, \emph{abstract argumentation} theory \cite{bondarenko1997}. 
%
Defeasible reasoning, in particular, is widely adopted in \emph{AI \& law} applications, to represent the complex intertwining of legal norms, often overlapping among each other, possibly from different, non-coherent sources.
%
Abstract argumentation theory, in its turn, is concerned with the formalisation and implementation of methods for rationally resolving disagreements, providing a general approach for modelling conflicts between arguments, and a semantics to establish if an argument can be acceptable or not.

Belief-desire-intention (BDI) logic is a kind of modal logic used for formalising, validating, and designing cognitive \emph{agents}---typically, in the \emph{multi-agent systems} (MAS) context.
%
A cognitive agent is an entity consisting of 
\emph{(i)} a belief base storing the agent's \emph{beliefs}, i.e. what the agent knows about the world, itself, and other agents;
\emph{(ii)} a set of \emph{desires} (or goals), i.e. the proprieties of the world the agent wants to eventually become true;
\emph{(iii)} a \emph{plan} library, encapsulating the agent's procedural knowledge (in the form of plans) aimed at making some goals become true; and
\emph{(iv)} a set of \emph{intentions}, storing the states of the plans the agent is currently enacting as an attempt to satisfy some desires.
%
All such data usually consist of first-order formulas.
%
Then, the dynamic behaviour of a BDI agent is driven by either internal (updates to the belief-base or changes in the set of desires) or external (perceptions or messages coming from the outside) events, which may cause new intentions to be created, or current intentions to be dropped.
%
By suitably capturing the revision of beliefs, and supporting the concurrent execution of goal-oriented computations, BDI architectures overcome critical issues of ``classical'' logic-based technologies -- \emph{concurrency} and \emph{mutability} -- in a sound way.
%
Overall, BDI architecture leads to a clear and intuitive design, where the underlying BDI logic provides for the formal background.
%
Among the frameworks rooted on a BDI approach, let us mention the AgentSpeak(L) \cite{Rao96} abstract language and its major implementation, namely Jason, Structured Circuit Semantics \cite{lee1994}, Act Plan Interlingua \cite{huber1999-jam}, JACK \cite{howden2001}, and dMARS---a platform for building complex, distributed, time-critical systems in C++ \cite{dmars}.

\subsection{Inference}

Deduction, Abduction, Induction, Probabilistic, etc.

\subsection{Logic Programming}

\cite{logictech-information11}
\cite{lptech4mas-aamas2021}
\cite{lptech4mas-jaamas35}
\cite{Korner2020HistoryFuturePrologTPLP}

\section{Sub-symbolic Reasoning}

\subsection{Neuro-Symbolic Computation}

\subsection{Knowledge Graph Embedding}

\chapter{Explaining AI via Symbolic Knowledge}

Despite the open philosophical issues, it is undeniable that AI and ML are nowadays becoming more and more intertwined with a growing number of aspects of people's every day life \cite{helbing2019, elliott2019}. 
%
In fact, more and more decisions are delegated by humans to software agents whose intelligent behaviour is not the result of some skilled developer endowing it with some clever code, but rather the consequence the agents' capability of learning, planning, or inferring what to do from data.

In spite of the large adoption, intelligent machines whose behaviour is the result of automatic synthesis / learning procedures are difficult to trust for most people---in particular when they are not expert in the field.
%
This is especially true for agents leveraging on machine or deep learning based techniques, often producing models whose internal behaviour is opaque and hard to explain for their developers too.

There, agents often tend to accumulate their knowledge into \emph{black-box} predictive models which are trained through ML or DL.
%
Broadly speaking, the ``black-box'' expression is used to refer to models where knowledge is not explicitly represented -- such as in NN, support vector machines (SVM), or random forests --, and it is therefore difficult, for humans, to understand what a black-box actually knows, or what leads to a particular decision.

Such difficulty in understanding black-boxes content and functioning is what prevents people from fully trusting -- and thus accepting -- them.
%
In several contexts, such as the medical or financial ones, it is not sufficient for intelligent agents to output bare decisions, since, for instance, ethical and legal issues may arise. 
%
An \emph{explanation} for each decision is therefore often desirable, preferable, or even required.  
%
Furthermore, it may happen for instance that black-boxes \emph{silently} learn something wrong (e.g., Google image recognition software that classified black people as gorillas \cite{fourcade2017, crawford2016artificial}), or something right, but in a biased way (like  the ``background bias'' problem, causing for instance husky images to be recognised only because of their snowy background \cite{RibeiroSG16}).

To tackle such trust issues, the \emph{eXplainable Artificial Intelligence} (XAI) research field has recently emerged, and a comprehensive research road map has been proposed by DARPA \cite{darpa2016-xai}, targeting the themes of explainability and interpretability in AI -- and in particular ML -- as a challenge of paramount importance in a world where AI is becoming more and more pervasively adopted.
%
There, DARPA reviews the main approaches to make AI either more interpretable or \emph{a posteriori} explainable, it categorises the many currently available techniques aimed at building meaningful interpretations or explanations for black-box models, it summarises the open problems and challenges, and it provides a reference framework for the researchers interested in the field.

Broadly speaking, research efforts in the field of XAI are focused on achieving key properties in AI, such as \emph{interpretability}, \emph{transparency}, \emph{explainability}, \emph{accountability}, and \emph{trustworthiness}.
%
Unfortunately, such goals are still far from being reached.
%
For instance, as pointed out in \cite{Lipton18}, the aforementioned properties are still lacking a formal and agreed-upon definition.
%
Some authors \cite{Rudin2019} propose to strengthen the adoption of most \emph{interpretable} (i.e. algorithmically transparent) predictive models -- such as generalised linear models or decision trees --, while others seek new ways to produce \emph{post-hoc} explanations capable of tackling even most opaque predictors, such as NN.
%
However, as demonstrated by the comprehensive survey produced by Guidotti et al. \cite{GuidottiMRTGP19}, most works only target classification problems, and they rarely take wider properties -- such as accountability and trustworthiness -- into account.

\sidenote{Spostare il capoverso sotto in un posto più opportuno}
Several works, among the various springs of AI, proposed to extract symbolic knowledge from sub-symbolic models.
%
As witnessed by a number of surveys \cite{GuidottiMRTGP19, GarcezBRFHIKLMS15, AndrewsDT95, GarcezBG01} and works on the topic \cite{BolognaH18, BolognaH16, FrosstH17, JohanssonN09, KrishnanSB1999, HruschkaE2006, ZhouZYS1983, CravenS95, AugastaK12, SatoT2002, KahramanliA09a} -- some of which are from the 80s or the 90s -- the potential of symbolic knowledge \emph{extraction} is well understood, despite not being subject to hype.
%
Unfortunately, a comprehensive and general framework tackling such problem in a general way is still missing.

\note{Armonizzare quanto segue con ciò che precede}

Since the adoption of interpretable predictors usually comes at cost of a lower potential in terms of predictive performance, \emph{explanations} are the newly preferred way for providing understandable predictions without necessarily sacrificing accuracy.
%
The idea, and the main goal of XAI is to create intelligible and understandable explanations for uninterpretable predictors \emph{without} replacing or modifying them.
%
Thus explanations are built through a number of heterogeneous techniques, broadly referred to as \emph{explanators} \cite{GuidottiMRTGP19}---just to cite some, \emph{decision rules} \cite{Augasta2012}, \emph{feature importance} \cite{tolomei2017interpretable}, saliency masks \cite{FongV17}, sensitivity analysis \cite{SundararajanTY17}, etc.

The state of the art for explainability currently recognises two main sorts of explanators, namely, either local or global. 
%
While \emph{local} explanators attempt to provide an explanation for each particular prediction of a given predictor $p$, the \emph{global} ones attempt to provide an explanation for the predictor $p$ as a whole.
%
In other words, local explanators provide an answer to the question ``why does $p$ predict $\mathbf{y}$ for the input $\mathbf{x}$?'' -- such as the LIME technique presented in \cite{RibeiroSG16} --, whereas global explanators provide an answer to the question ``how does $p$ build its predictions?''---such as decision rules.

In spite of the many approaches proposed to explain black boxes, some important scientific questions still remain unanswered. 
%
One of the most important open problems is that, until now, there is no agreement on what an explanation is. 
%
Indeed, some approaches adopt as explanation a set of rules, others a decision tree, others rely on visualisation techniques \cite{GuidottiMRTGP19}. 
%
Moreover, recent works highlight the importance for an explanation to guarantee some properties, e.g., soundness, completeness, and compactness \cite{GuidottiMRTGP19}. 

% \note{Probabilmente si può tagliare da qui in poi}

% This is why our proposal aims at integrating sub-symbolic approaches with symbolic ones.
% %
% To this end, DT can be exploited as an effective bridge between the symbolic and sub-symbolic realms.
% %
% In fact, DT can be easily \emph{(i)} built from an existing sub-symbolic predictor, and \emph{(ii)} translated into symbolic knowledge -- as it is shown in the reminder of this paper -- thanks to their rule-based nature.

% Decision trees are an interpretable family of predictors that have been proposed as a \emph{global} means for explaining other, less interpretable, sorts of black-box predictors \cite{TrattnerPR2019,BastaniKB17}---such as neural networks \cite{CravenS95}.
% %
% The main idea behind such an approach is to build a DT approximating the behaviour of a given predictor, possibly, by only considering its inputs and its outputs.
% %
% Such approximation essentially trades off predictive performance with interpretability.
% %
% In fact, the structure of such a DT would then be used to provide useful insights concerning the original predictor inner functioning. 

% Describing the particular means for extracting DT from black-boxes is outside the scope of this paper.
% %
% Given the vast literature on the topic -- e.g., consider reading \cite{GuidottiMRTGP19,AndrewsDT95} for an overview or \cite{CravenS95,JohanssonN09,FrosstH17} for a practical examples -- we simply assume an extracted DT is available and it has an high \emph{fidelity}---meaning that the loss in terms of predictive performance is low, w.r.t. the original black-box.
% %
% In fact, whereas there exist several works focussing on how to synthesise DT out of black-box predictors, no attention is paid to merging them with symbolic approaches, which can play a key role in enhancing the interpretability and explainability of the system.
% %
% In this paper we focus on such a matter.

% We believe that a logical representation of DT may be interesting and enabling for further research directions.
% %
% For instance, as far as explainability is concerned, we show how logic-translated DT can be used to both navigate the knowledge stored within the corresponding predictors -- thus acting as \emph{global} explanators --, and produce \emph{narrative} explanations for their predictions---thus acting as \emph{local} explanators. 
% %
% Note that the restriction on the DT representation makes it easy to map DT onto logical clauses, since DT are finite and with a limited expressivity (if / else conditions).

%===============================================================================
\section{XAI Background}\label{sec:background}
%===============================================================================

Most intelligent systems (IS) today leverage on \emph{numerical} predictive models which are trained from data through ML.
%
The reason for such a wide adoption is easy to understand.
%
We live in an era where the availability of data is unprecedented, and ML algorithms make it possible to semi-automatically detect useful statistical information hidden into such data.
%
Information, in turn, supports decision making, monitoring, planning, and forecasting in virtually any human activity where data is available.

However, ML is not the silver bullet.
%
Despite the increased predictive power, ML comes with some well-known drawbacks which make it perform poorly in some use cases.
%
One blatant example is algorithmic \emph{opacity}---that is, essentially, the difficulty of human mind in \emph{understanding} how ML-based IS function or compute their outputs.
%
Such difficulty is a serious issue in all those contexts where human beings are liable for their decision or must provide some sort of \emph{explanation} for it---even if the decision has been supported by some IS.
%
For instance, think about a doctor willing to motivate a serious, computer-aided diagnosis, or, a bank employee in need of explaining to a customer why his/her profile is inadequate for a loan.
%
In all contexts, ML is at the same time an enabling -- as it aids the decision process by automating it -- and a limiting factor---as opacity prevents human awareness of \emph{how} the decision process works.

Opacity is why ML predictors are also referred to as \emph{black boxes} into the literature.
%
The ``black box'' expression refers to models where knowledge is not explicitly represented \cite{Lipton18}. 
%
The lack of some explicit, symbolic representation of knowledge is what makes it hard for humans to \emph{understand} the functioning of black boxes, and why they led to suggest or undertake a given decision.
%
Obviously, troubles in understanding black-box content and functioning prevents people from fully trusting -- therefore accepting -- them. 
%
To make the picture even more complex, current regulations such as the GDPR \cite{gdpr-voigt2017} are starting to recognise the citizens' \emph{right to explanation} \cite{explanation-aimag38}---which implicitly requires IS to eventually become \emph{understandable}.
%
In fact, understanding IS is essential to guarantee algorithmic fairness, to identify potential bias/problems in the training data, and to ensure that IS perform as designed and expected. 

%%%%
\begin{figure}
	\centering
	\includegraphics[width=.5\linewidth]{figures/interpretability-performance-tradeoff}
	\caption{Interpretability/performance trade-off for some common sorts of black-box predictors}
	\label{fig:tradeoff}
\end{figure}
%%%%

Unfortunately, the notion of understandability is neither standardised nor systematically assessed, yet.
%
At the same time, there is no consensus on what exactly providing an \emph{explanation} should mean when decisions are supported by a black box.
%
However, several authors agree that not all black boxes are equally \emph{interpretable}---meaning that some black boxes are easier to understand than others for our mind.
%
For example, \Cref{fig:tradeoff} is a common way to illustrate the differences in black-box interpretability.

Even though informal -- as pointed on in \cite{Rudin2019}, given the lack of way to measure ``interpretability'' -- \Cref{fig:tradeoff} effectively express why more research is need on understandability.
%
In fact, the image essentially states how the better performing black boxes are also the less interpretable ones.
%
This is a problem in practice since only rarely predictive performances can be sacrificed in favour of an higher degree of interpretability.

To tackle such issues, the \emph{eXplainable AI} (XAI henceforth) research field has recently emerged.
%
Among the many authors and organisations involved in the topic, DARPA has proposed a comprehensive research road map \cite{darpa2016-xai} which reviews the main approaches to make black boxes more understandable.
%
There, DARPA categorises the many currently available techniques aimed at building meaningful interpretations or explanations for black-box models, it summarises the open problems and challenges, and it provides a successful reference framework for the researchers interested in the field.
%
Unfortunately, in spite of the great effort in defining terms, objects, and methods for the research line, a clear definition of fundamental notions such as \emph{interpretation} and \emph{explanation} is still missing.

%-------------------------------------------------------------------------------
\subsection{Related works}\label{ssec:related}
%-------------------------------------------------------------------------------

Notions such as explanation, interpretation, transparency, etc, are mentioned, introduced, or informally defined in several works.
%
However, a coherent framework has not emerged yet.

In this subsection we recall some of the main contributions from the literature where the concepts of explanation and interpretation -- or any variant of theirs -- are discussed.
%
Our goal here is to highlight the current lack of consensus on the meaning of such terms, for which we propose a possible, unambiguous alternative in the next sections.

Similarly to what we do here, Lipton \cite{Lipton18} starts his discussion by recognising how most definition of ML interpretability are often inconsistent and underspecified.
%
In his clarification effort, Lipton essentially maps interpretability on the notion of \emph{transparency}, and explanation on the notion of \emph{post-hoc} interpretation.
%
Then, he enumerates and describes the several possible variants of transparency, that are 
%
\begin{enumerate*}[label=\emph{(\roman{*})}]
	\item simulatability -- i.e., the \emph{practical} possibility, for a human being, to ``contemplate the entire model at once'' and simulate its functioning on some data  -- which characterises, for instance, generalised linear models;
	
	\item decomposability -- i.e., the possibility, for the model to be decomposed in elementary parts whose functioning is intuitively understandable for humans and helpful in understanding the whole model -- which characterises, for instance, decision trees; and
	
	\item algorithmic transparency -- i.e., the possibility, for a human being, to intuitively understand how a given learning algorithm, or the predictors it produces, operate -- which characterises, for instance, k-nearest-neighbours techniques.
\end{enumerate*}
%
Similarly, \emph{post-hoc} interpretability is defined as an approach where some information is extracted from a black box in order to ease its understanding.
%
Such information have not necessarily to expose the internal functioning of the black box.
%
As stated in the paper: ``examples of post-hoc interpretations include the verbal explanations produced by people or the saliency maps used to analyze deep neural networks''.

Conversely, Besold et al. \cite{BesoldU2018} discuss the notion of explanation at a fundamental level.
%
There, the authors provide a nice philosophical overview on such topic, concluding that ``explanation is an epistemological activity and explanations are an epistemological accomplishment---they satisfy a sort of epistemic longing, a desire to know something more than we currently know. Not only do they satisfy this desire to know, they also provide the explanation-seeker a direction of action that they did not previously have''.
%
Then they discuss the topic of explanation in AI from an historical perspective.
%
In particular, when focussing on ML, they introduce the following classification of IS systems:
%
\begin{enumerate*}[label=\emph{(\roman{*})}]
	\item opaque systems -- i.e., black boxes acting as oracles where the logic behind predictions is not observable or understandable --,
	
	\item interpretable systems -- i.e., white boxes whose functioning is understandable to humans, also thanks to expertise, resources, or tools --, and
	
	\item comprehensible systems---i.e., ``systems which emit \emph{symbols} along with their outputs, allowing the user to relate properties of the input to the output''.
\end{enumerate*}
%
According to this classification, while interpretable systems can be inspected to be understood -- thus letting observer draw their explanations by themselves--, comprehensible systems must explicitly provide a symbolic explanation of their functioning.
%
The focus there is thus on \emph{who} produces explanations, rather than \emph{how}.

In \cite{DoshiVelezK2017}, interpretability of ML systems is defined as ``the ability to explain or to present in understandable terms to a human''.
%
Interpretations and explanations are therefore collapsed in this work, as confirmed by the authors using the two terms interchangeably.
%
The reminder of that paper focuses 
\begin{enumerate*}[label=\emph{(\roman{*})}]
	\item on identifying under which circumstances interpretability is needed in ML, and
	\item how to assess the quality of some explanation.
\end{enumerate*}

The survey by Guidotti et al. \cite{GuidottiMRTGP19} is a nice entry point to explainable ML.
%
It consists of an exhaustive and recent survey overviewing the main notions, goals, problems, and (sub-)categories in this field and it encompasses a taxonomy of existing approaches for ``opening the black box''---which may vary a lot depending on the sort of data and the family of predictors at hand.
%
There, the authors define the verb to interpret as the act of ``providing some meaning of explaining and presenting in understandable terms some concepts'', borrowing such a definition from the Merriam-Webster\footnote{\url{https://www.merriam-webster.com/dictionary/interpret}} dictionary.
%
Consequently, they define interpretability as ``the ability to explain or to provide the meaning in understandable terms to a human''---a definition they again borrow from \cite{DoshiVelezK2017}.
%
So, in this case as well the notions of \emph{interpretation} and \emph{explanations} are collapsed.

In \cite{Rudin2019}, Rudin does not explicitly provide a definition for explainability or interpretability, and she refers about interpretable or explainable ML almost interchangeably.
%
However, she states some interesting properties about \emph{interpretability}, which influenced our work.
%
In particular, she acknowledges that ``interpretability is a domain-specific notion''.
%
Furthermore, she links interpretability of information with its complexity -- and, in particular, its \emph{sparsity} --, as the amount of cognitive entities the human mind can at one is very limited ($\sim 7 \pm 2$ according to \cite{numberseven-psyrev63}). 
%
As far as explainability is concerned, apparently, Rudin adopts a \emph{post-hoc} perspective similar to the one in \cite{Lipton18}, as she writes ``an explanation is a separate model that is supposed to replicate most of the behaviour of a black box''.
%
In the reminder of the paper, the author argues how the path towards interpretable ML steps through a wider adoption of inherently interpretable predictors -- such as generalised linear models or decision trees -- instead of the relying on \emph{post-hoc} explanations which do not reveal what is inside black boxes---thus preventing their full understanding.

Finally, the recent article by Rosenfeld et al. \cite{RosenfeldR2019} is similar in its intents to our current work.
%
There, the authors attempt to formally define what explanation and interpretation respectively are in the case of ML-based classification. 
%
However, their work differs from ours in several ways.
%
In particular, they define interpretation and explanation differently from what we do.
%
In fact, according to the authors, ``interpretation'' is a function mapping data, data schemes, and predictors to some representation of the predictors internal logic, whereas ``explanation'' is defined as ``the human-centric objective for the user to understand'' a predictor using the aforementioned interpretation function.
%
Other notions are formally defined into the paper, such as for instance, 
%
\begin{enumerate*}[label=\emph{(\roman{*})}]
	\item explicitness, 
	\item fairness, 
	\item faithfullness,
	\item justification, and
	\item transparency.
\end{enumerate*}
%
Such concepts are formally defined in terms of the aforementioned interpretation and explanation functions.
%
The reminder of the paper then re-interprets the field of XAI in terms of all the notions mentioned so far. 

\section{Explanation vs. Interpretation}\label{sec:basics}

This section introduces the preliminary notions, intuitions, and notations we leverage upon in \cref{ssec:framework} and subsequent sections, in order to formalise our abstract framework for agent-based explanations.
%
We start by providing an intuition for the notion of \emph{interpretation}, and, consequently, for the \emph{act} of interpreting something.
%
Accordingly, we provide an intuition for the property of ``being interpretable'' as well, stressing its comparative nature.
%
Analogously to what we did with \emph{interpretation}, we then provide intuitions for terms such as \emph{explanation} and its derivatives.

%---
\paragraph{About interpretation.}
%---

Taking inspiration from the field of Logics, we define the \emph{act} of ``interpreting'' some object $X$ as the activity performed by an agent $A$ -- either human or software -- assigning a \emph{subjective} meaning to $X$.
%
Such meaning is what we call \emph{interpretation}.
%
Roughly speaking, an object $X$ is said to be  \emph{interpretable} for an agent $A$ if it is \emph{easy} for $A$ to draw an interpretation for $X$---where ``easy'' means $A$ requires a low \emph{computational} (or \emph{cognitive}) effort to understand $X$.
%
For instance, consider the case of road signs, which contain symbols instead of scripts to be easily, quickly, and intuitively interpretable.

We model such intuition through a function $I_A(X) \mapsto [0, 1]$ providing a \emph{degree of interpretability} -- or simply interpretability, for short -- for $X$, in the eyes of $A$.
%
The value $I_A(X)$ is not required to be directly observable or measurable in practice, since agents' mind may be inaccessible in most cases.
%
This is far from being an issue, since we are not actually interested in the absolute value of $I_A(X)$, for some object $X$, but rather we are interested in being able to order different objects w.r.t.\ their subjective interpretability.
%
For instance, we write $I_A(X) > I_A(Y)$, for two objects $X$ and $Y$, meaning that the former is more interpretable than the latter, according to $A$.
%
For example, consider the case of a neural network and a decision tree, both trained on the same examples to solve the same problem with similar predictive performances.
%
Both objects may be represented as graphs.
%
However, it is likely for a human observer to see the decision tree as more interpretable---as their nodes bring semantically meaningful, high-level information.

Summarising, we stress the subjective nature of interpretations, as agents assign them to objects according to their State of Mind (SoM) \cite{PremackW1978} and background knowledge, and they need not be formally defined any further.

%%%%
\begin{figure}
    \centering
    \includegraphics[width=.5\linewidth]{figures/framework.pdf}
    \caption{Explanation vs. Interpretation: a simple framework}
    \label{fig:framework}
\end{figure}
%%%%

%---
\paragraph{About explanation.}
%---
We define ``explaining'' as the activity of producing a more interpretable object $X'$ out of a less interpretable one, namely $X$, performed by agent $A$.
%
More formally, we define \emph{explanation} as a function $E(X) \mapsto X'$ mapping objects into other objects, possibly, in such a way that $I_A(X') > I_A(X)$, for some agent $A$.
%
The simple framework described so far is summarised in \cref{fig:framework}.

Notice that human beings tend to collapse into the concept of ``explanation'' the whole sequence of steps actually involving both explaining and interpreting, according to our framework.
%
This happens because, if the explained object $X'$ is as interpretable for the listening agent $B$ as it is for the explaining agent $A$, then both $A$ and $B$ are likely to be satisfied with $X'$.
%
Conversely, it may also happen the explanation $E$ adopted by $A$ produces an object $X'$, which is more interpretable than $X$ for $A$ but not for $B$.
%
Similarly to how two persons would handle such an unpleasant situation, we envision that interaction and communication may be adopted to break such \emph{impasses} in multi-agent systems.

In the following sections, we develop such an idea, describing how our simple framework could be extended to support ML-based intelligent systems.

%-------------------------------------------------------------------------------
\subsection{A conceptual framework for XAI}\label{ssec:framework}
%-------------------------------------------------------------------------------

In AI several tasks can be reduced to a functional model $M: X \rightarrow Y$ mapping some input data $X \subseteq \mathcal{X}$ from an input domain $\mathcal{X}$ into some output data $Y \subseteq \mathcal{Y}$ from an output domain $\mathcal{Y}$.

In the following, we denote as $\mathcal{M}$  the set of all \emph{analogous} models $M': X \rightarrow \mathcal{Y}$, which attempts to solve the same problem on the same input data---usually, in (possibly slightly) different ways.
%
For instance, according to this definition, a decision tree and a neural network, both trained on the same data-set to solve the same classification problem with similar accuracies, are analogous---even if they belong to different families of predictors.

At a very high abstraction level, many tasks in AI may be devoted to compute, for instance:
%
\begin{itemize}
    \item the best $M^* \in \mathcal{M}$, given $X \subseteq \mathcal{X}$ and $Y \subseteq \mathcal{Y}$ (e.g. supervised ML),
    \item the best $M^*$ and $Y$, given $X$ (e.g.\ unsupervised ML),
    \item the best $Y^*$, given $X$ and $M$ (e.g.\ informed/uninformed search),
    \item the best $X^*$, given $Y$ and $M$ (e.g.\  abduction, most likely explanation), etc 
\end{itemize}
%
according to some goodness criterion which is specific for the task at hand.

In the reminder of this section, we discuss how explanation may be defined as a function searching or building a -- possibly more interpretable -- model w.r.t.\ the one to be explained.
%
For this process to even make sense, of course, we require the resulting model to be not only analogous to the original but also similar in the way it behaves on the same data.
%
We formalise such a concept through the notion of \emph{fidelity}.

Let $M, M' \in \mathcal{M}$ be two analogous models.
%
We then say $M$ has a \emph{locally} good \emph{fidelity} w.r.t.\ $M'$ and $Z$ if and only if $\Delta f(M(Z), M'(Z)) < \delta$ for some arbitrarily small threshold $\delta \geq 0$ and for some subset of the input data $Z \subset X$.
%
There, $\Delta f : 2^\mathcal{Y} \times 2^\mathcal{Y} \rightarrow \mathbb{R}_{\geq 0}$ is a function measuring the performance \emph{difference} among two analogous models.

%%%%
\begin{figure}
    \centering
    \includegraphics[width=.5\linewidth]{figures/local.pdf}
    \caption{Local explanation and interpretation of model $M$}
    \label{fig:local}
\end{figure}
%%%%

%---
\paragraph{Local interpretations.}
%---

When an observer agent $A$ is \emph{interpreting} a model $M$ behaviour w.r.t.\ some input data $Z \subseteq X$, it is actually trying to assign a subjective interpretability value $I_A(R)$ to some representation $R = r(M, Z)$ of choice, aimed at highlighting the behaviour of $M$ w.r.t.\ the data in $Z$.
%
There, $r : \mathcal{M} \times 2^\mathcal{X} \rightarrow \mathcal{R}$ is \emph{representation means}, i.e., a function mapping models into \emph{local} representations w.r.t.\ a particular subset of the input domain, whereas $\mathcal{R}$ is the set of model representations.
%
For instance, in the case $M$ is a classifier, $R$ may be a graphical representation of (a portion of) the decision boundary/surface for a couple of input features.

There may be more or less interpretable \emph{representations} of a particular model for the same observer $A$.
%
Furthermore, representations may be either global or local as well, depending on whether they represent the behaviour of the model for the whole input space, or for just a portion of it.
%
For example, consider the case of a plot showing the decision boundary of a neural network classifier.
%
This representation is likely far more interpretable to the human observer than a graph representation showing the network structure, as it synthesise the global behaviour of the network concisely and intuitively.
%
Similarly, saliency maps are an interpretable way to \emph{locally} represent the behaviour of a network w.r.t. some particular input image.
%
So, a way for easing interpretation for a given model behaviour w.r.t.\ a particular sort of inputs is about looking for the right representation in the eyes of the observer.

%---
\paragraph{Local explanations.}    
%---

Conversely, when an observer $A$ is \emph{explaining} a model $M$ w.r.t.\ some input data $Z \subseteq X$, it is actually trying to produce a model $M' = E(M, Z)$ through some function $E: \mathcal{M} \times 2^\mathcal{X} \rightarrow \mathcal{M}$.
%
In this case, we say $M'$ is a \emph{local explanation} for $M$ w.r.t.\ to $Z$.
%
We also say that $M'$ is produced through the explanation strategy $E$.

Furthermore, we define an explanation $M'$ as \emph{admissible} if it has a valid fidelity w.r.t.\ the original model $M$ and the data in $Z$---where $Z$ is the same subset of the input data used by the explanation strategy.
%
More precisely, we say $M'$ is $\delta$-admissible in $Z$ w.r.t.\ $M$ if $\Delta f(M(Z), M'(Z)) < \delta$.

Finally, we define an explanation $M'$ as \emph{clear} for $A$, in $Z$, and w.r.t.\ the original model $M$, if there exists some representation $R' = r(M', Z)$ which is more interpretable than the original model representation $R$.
%
More precisely, we say $M'$ is $\varepsilon$-clear for $A$, in $Z$, and w.r.t $M$ if $I_A(R') - I_A(R) > \varepsilon$ for some arbitrarily big threshold $\varepsilon > 0$.

Several \emph{explanations} may actually be produced for the same model $M$.
%
For each explanation, there may be again more or less interpretable \emph{representations}.
%
Of course, explanations are useful if they ease the seek for more interpretable representations.
%
Thus, providing an explanation for a given model behaviour w.r.t.\ a particular class of inputs is about creating \emph{ad-hoc} metaphors aimed at easing the observer's understanding.

%%%%
\begin{figure}
    \centering
    \includegraphics[width=.5\linewidth]{figures/global.pdf}
    \caption{Global explanation and interpretation of model}
    \label{fig:global}
\end{figure}
%%%%

%---
\paragraph{Global / local explanations.}
%---

The theoretical framework described so far -- which is graphically synthesised in \cref{fig:local} -- is aimed at modelling \emph{local} interpretations and explanations, that are, the two means an explanator agent may exploit in order to make AI tasks' \emph{outcomes} more understandable in the eyes of some explanee. 

Conversely, when the goal is not to understand some model outcome, but the model itself, from a \emph{global} perspective -- or, equivalently, when the goal is to understand the model outcome w.r.t the whole set of input data $X$ --, the theoretical framework described so far is simplified as shown in \cref{fig:global}, where the dependency on the input data is omitted from functions $E$, $\Delta f$, and $r$.
%
This is possible because we consider the global case as a particular case of the local one, where $Z \equiv X$.

Finally, we remark that the case where a model $M$ is to be understood on a single input-output pair, say $x$ and $y = M(x)$, is simply captured by the aforementioned local model, through the constraint $Z = \{ x \}$ and $M(Z) = \{ y \}$.

%-------------------------------------------------------------------------------
\subsection{Discussion}
%-------------------------------------------------------------------------------

Our framework is deliberately abstract in order to capture a number of features we believe to be essential in XAI.
%
First of all, our framework acknowledges -- and properly captures -- the orthogonality of interpretability w.r.t.\ explainability. 
%
This is quite new, indeed, considering that most authors tend to use the two concepts as if they were equivalent or interchangeable.

Furthermore, our framework explicitly recognises the \emph{subjective} nature of interpretation, as well as the subtly \emph{objective} nature of explanation.
%
Indeed, interpretation is a subjective activity directly related to agents' perception and SoM, whereas explanation is an epistemic, computational action which aims at producing a high-fidelity model.
%
The last step is objective in the sense that it does not depend on the agent's perceptions and SoM, thus being reproducible in principle.
%
Of course, the \emph{effectivess} of an explanation is again a subjective aspect.
%
Indeed, a clear explanation (for some agent) is a more interpretable variant of some given model--- thus, the subjective activity of interpretation is again implicitly involved.

The proposed framework also captures the importance of representations.
%
This is yet another degree of freedom that agents may exploit in their seek for a wider understandability of a given model.
%
While other frameworks consider interpretability as an intrinsic property of AI models, we stress the fact that a given model may be represented in several ways, and each representation may be interpreted differently by different agents.
%
As further discussed in the remainder of this paper, this is far from being an issue.
%
This subjectivity is deliberate, and it is the starting point of some interesting discussions.

Finally, our framework acknowledges the global/local duality of both explanation and interpretation, thus enabling AI models to be understood either general or with respect to a particular input/output pair.

%-------------------------------------------------------------------------------
\subsection{Practical remarks}
%-------------------------------------------------------------------------------

The ultimate goal of our framework is to provide a general, flexible, yet minimal framework describing the many aspects concerning AI understandability in the eyes of a \emph{single} agent.
%
We here illustrate several practical issues affecting our framework in practice, and further constraining it.

According to our conceptual framework, a \emph{rational} agent seeking to understand some model $M$ (or make it understandable) may either choose to elaborate on the \emph{interpretation axis} -- thus looking for a (better) representation $R$ of $M$ -- or it can elaborate on the \emph{explainability axis}---thus producing a novel, high fidelity model $M'$, coming with a representation $R'$ which is more interpretable than the original one (i.e., $R$).

Notice that, in practice, the nature of the model constrains the set of admissible representations.
%
This means that a rational agent is likely to exploit both the explanation and interpretation axes in the general case---because novel representations may become available through an explanation.
%
we argue and assume that each family of AI models comes with just a few \emph{natural} representations.
%
Because of this practical remark, we expect that, in real-world scenarios, an agent seeking for understandability is likely to ``work'' on both the interpretation and the explanation axes.

For instance, consider decision trees, which come with a natural representation as a tree of subsequent choices leading to a decision.
%
Conversely, neural networks can either be represented as graphs or as algebraic combinations of tensors.
%
In any case, neural network models are commonly considered less interpretable than other models.
%
In such situation, a rational agent willing to make a neural network more understandable may choose to combine decision trees extraction (explanation) -- possibly focusing on methods from the literature \cite{AndrewsDT95,xailp-woa2019} -- to produce a decision tree whose tree-like structure (representation) could be presented to the human observer to ease his/her interpretation.
%
The decision-tree like representation is not ordinarily available for neural networks, but it may become available provided that an explanation step is performed.

Another interesting trait of our framework concerns the semantics of clear explanations.
%
The current definition requires explanation strategies to consume a model $M$ with a given representation $R$ and to produce a high-fidelity model $M'$ for which a representation $R'$ exists, which is more interpretable than $R$.
%
Several semantics may fit this definition.
%
This is deliberate, since different semantics may come with different computational requirements, properties, and guarantees.
%
For instance, one agent may be interested in finding the \emph{best} explanation---that is, the one for which \emph{each} representation is more interpretable than the most interpretable representation of the original model.
%
Similarly, in some cases, it may be sufficient -- other than more feasible -- to find an \emph{admissible} explanation---that is, a high-fidelity model for which \emph{some} representation exists that is more interpretable than \emph{some} representation of the original model. 
%
However, the inspection of the possible semantics and their properties falls outside the scope of this paper and is going to be considered as a future research direction.

\subsection{Assessment of the Framework}\label{sec:validation}

The abstraction level of the presented framework has also been conceived in order to capture most of the current state of the art.
%
Along this line, this section aims at validating the fitting of the existing contributions w.r.t.\ the framework presented in section \cref{ssec:framework}: if our framework is expressive enough, it should allow most (if not all) existing approaches to be uniformly framed, to be easily understood and compared.
%
To this end, we leverage on the work by Guidotti et al.\ \cite{GuidottiMRTGP19}, where the authors perform a detailed and extensive survey on the state-of-the-art methods for XAI, by categorising the surveyed methods according to an elegant taxonomy.
%
Thus, hereafter, we adopt their taxonomy as a reference for assessing our framework.

The taxonomy proposed by Guidotti et al.\ essentially discriminates among two main categories of XAI methods.
%
These are the ``transparent box design'' and the ``black-box explanation'' categories. 
%
While the former category is not further decomposed, the latter comes with three more sub-categories, such as ``model explanation'', the ``outcome explanation'', and the ``model inspection''.
%
Notice that, despite the authors' definition of ``explanation'' does not precisely match the one proposed in this paper, we maintained the original categorisation.

The remainder of this section navigates such a taxonomy accordingly, by describing how each (sub-)category -- along with the methods therein located -- fits our abstract framework.

%-------------------------------------------------------------------------------
\subsubsection{Model explanation}
%-------------------------------------------------------------------------------

The mapping of the methods classified as part of the ``model explanation'' sub-category into our framework is seamless.
%
Hence, it can be defined as follows:
%
\begin{itemize}
    \item[] Let $M$ be a sub-symbolic classifier whose internal functioning representation $R$ is poorly interpretable in the eyes of some explanee $A$, and let $E(\cdot)$ be some \emph{global} explanation strategy.
    %
    Then, the model explanation problem consists of computing some \emph{global} explanation $M' = E(M)$ which is $\delta$-admissible and $\varepsilon$-clear w.r.t.\ to $A$, for some $\delta, \varepsilon > 0$. 
\end{itemize}
%
For instance, according to Guidotti et al., possible sub-symbolic classifiers are neural (possibly deep) networks, support vector machines, and random forests.
%
Conversely, explanation strategies may consist of algorithms aimed at 
%
\begin{enumerate*}[label=\emph{(\roman{*})}]
    \item extracting decision trees/rules out of sub-symbolic predictors and the data they have been trained upon, 
    \item compute feature importance vectors,
    \item detecting saliency masks,
    \item detecting partial dependency plots, etc. 
\end{enumerate*}

In our framework, all the algorithms mentioned above can be described as \emph{explanation strategies}. 
%
Such mapping is plausible given their ability to compute an admissible, and possibly more explicit models out of black boxes and the data they have been trained upon.
%
However, it is worth to highlight that the clarity gain produced by such explanation strategies mostly relies on the implicit assumption that their output models come with a natural representation which is intuitively interpretable to the human mind.

%-------------------------------------------------------------------------------
\subsubsection{Outcome explanation}
%-------------------------------------------------------------------------------

Methods classified as part of the ``outcome explanation'' sub-category can be very naturally described in our framework as well.
%
In fact, it can be defined as follows:
%
\begin{itemize}
    \item[] Let $M$ be some sub-symbolic classifier whose internal functioning representation $R = r(M, Z)$ in some subset $Z \subset \mathcal{X}$ of the input domain is poorly interpretable to some explanee $A$, and let $E(\cdot, \cdot)$ be some \emph{local} explanation strategy.
    %
    Then, the outcome explanation problem consists of computing some \emph{local} explanation $M' = E(M, Z)$ which is $\delta$-admissible and $\varepsilon$-clear w.r.t.\ to $A$, for some $\delta, \varepsilon > 0$
\end{itemize}
%
Summarising, while input black boxes may still be classifiers of any sort, explanation, and explanation strategies differ from the ``model explanation'' case.
%
In particular, explanation strategies in this sub-category may rely on techniques leveraging on attention models, decision trees/rules extraction, or well-established algorithms such as LIME \cite{RibeiroSG16}, and its extensions---which are essentially aimed at estimating the contribution of every input feature of the input domain to the particular outcome of the black box to be explained.

Notice that the explanation strategies in this category are only required to be admissible and clear in the portion of the input space surrounding the input data under study.
%
Such a portion is implicitly assumed to be relatively small in most cases.
%
Furthermore, the explanation strategy is less constrained than in the global case, as it is not required to produce explanations elsewhere.

%-------------------------------------------------------------------------------
\subsubsection{Model inspection}
%-------------------------------------------------------------------------------

Methods classified as part of the ``model inspection'' sub-category can be naturally defined as follows:
%
\begin{itemize}
    \item[] Let $M$ be a sub-symbolic classifier whose available \emph{global} representation $R = r(M)$ is poorly interpretable to some explanee $A$, and let $r(\cdot), r'(\cdot)$ be two different representation means.
    %
    Then, the model inspection problem consists of computing some representation $R' = r'(M)$ such that $I_A(R') > I_A(R)$
\end{itemize}
%
Of course, solutions to the model inspection problem vary a lot depending on which specific representation means $r(\cdot)$ is exploited by the explanator, other than the nature of the data the black box is trained upon.
%
Guidotti et al.\ also provide a nice overview of the several sorts of representations means which may be useful to tackle the model inspection problem, like, for instance, sensitivity analysis, partial dependency plots, activation maximization images, tree visualisation, etc.

It is worth pointing out the capability of our framework to reveal the actual nature of the inspection problem.
%
Indeed, it clearly shows how this is the first problem among the ones presented so far, which only relies on the interpretation axis alone to provide understandability.

%-------------------------------------------------------------------------------
\subsubsection{Transparent box design}
%-------------------------------------------------------------------------------

Finally, methods classified as part of the ``transparent box design'' sub-category can be naturally defined as follows:
%
\begin{itemize}
    \item[] Let $X \subseteq \mathcal{X}$ be a dataset from some input domain $\mathcal{X}$, let $r(\cdot)$ be a representation means, and let $A$ be the explanee agent.
    %
    Then the transparent box design problem consists of computing a classifier $M$ for which a global representation $R = r(M, X)$ exists such that $I_A(R) > 1 - \delta$, for some $\delta > 0$
\end{itemize}
%
Although very simple, the transparent-box design is of paramount importance in XAI systems as it is the basic brick of most general explanation strategies.
%
Indeed, it may be implicit in the functioning of some explanation strategy $E$ to be adopted in some other model or outcome explanation problem.

For instance, consider the case of a local explanation strategy $E(M, X) \mapsto M'$.
%
In the general case, to compute $M'$, it relies on some input data $X$ and the internal of the to-be-explained model $M$.
%
However, there may be cases where the actual internal of $M$ are not considered by the particular logic adopted by $E$.
%
Instead, in such cases, $E$ may only rely on $X$ and the outcomes of $M$, which are $Y = M(X)$.
%
In this case, the explanation strategy $E$ is said \emph{pedagogical}---whereas in the general case it is said \emph{decompositional} (cf.\ \cite{AndrewsDT95}).

In other words, as made evident by our framework, the pedagogical methods exploited to deal with the model or outcome explanation problems must internally solve the transparent box design problem, as they must build an interpretable model out of some sampled data-set and nothing more.

\section{Symbolic Knowledge Extraction}

\cite{shallow2deep-extraamas2021}
\cite{xailp-woa2019}

\section{Symbolic Knowledge Injection}

\cite{nsc4xai-woa2020}

\part{How}
\label{part:how}

\chapter{The Role of Software}

\chapter{Technological State of the Art}

\cite{coordination-jlamp2020}

\section{Current State of Logic-Based Technologies}

\cite{lptech4mas-aamas2021}
\cite{lptech4mas-jaamas35}
\cite{logictech-information11}

\section{Current State of Machine Learning Technologies}

\section{Current State of XAI Technologies}

\cite{xaisurvey-ia14}

\chapter{Need for An Open Ecosystem for Logic-Based AI}

\cite{cco-softwarex-2021-2pkt}

\chapter{The 2P-Kt Ecosystem}

\cite{cco-softwarex-2021-2pkt}
\cite{kotlindsi4prolog-woa2020}

\chapter{Bridging Logic Programming and Data Processing}

\cite{2pkt-jelia2021}

\chapter{Bridging Logic Programming and Object Orientation}

\cite{cco-softwarex-2021-2pkt}
\cite{kotlindsi4prolog-woa2020}

\chapter{Bridging Logic Programming and Machine Learning}

Castiglio

\chapter{Bridging Logic Programming and XAI}

Psyke

\chapter{Enriching the Ecosystem}

\section{Probabilistic Logic Programming}

Jason

\section{Argumentation}

Pisano

\section{Inductive Logic Programming}

Speciale

\part{Who}
\label{part:who}

\cite{imagination-extraamas2021}
\cite{expectation-extraamas2021}

\chapter{Adding Control to Data via Agents}

\chapter{On the role of Interaction}

\cite{tusow-icccn2019}
\cite{respect-idc2017}
\cite{respectx-comsis15}

\chapter{Blockchain as the way to Trustworthiness}

\cite{bctcoord-bct4mas2018wi}
\cite{bctcoord-bct4mas2019}
\cite{bctcoordination-information11}
\cite{blockchain-goodtechs2018}
\cite{proactivesc-blockchain2019}
\cite{blockchainmas-applsci10}


%----------------------------------------------------------------------------------------
% BIBLIOGRAPHY
%----------------------------------------------------------------------------------------

\backmatter

\part*{}

% \nocite{*} % uncomment this to show all the reference in the .bib file
\bibliographystyle{alpha}
\bibliography{phd-thesis}


\end{document}